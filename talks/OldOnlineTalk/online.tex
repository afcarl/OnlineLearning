\documentstyle[12pt]{article}

\newcommand{\withpsfigs}[1]{#1}
\withpsfigs{% Psfig/TeX 
\def\PsfigVersion{1.9}
% dvips version
%
% All psfig/tex software, documentation, and related files
% in this distribution of psfig/tex are 
% Copyright 1987, 1988, 1991 Trevor J. Darrell
%
% Permission is granted for use and non-profit distribution of psfig/tex 
% providing that this notice is clearly maintained. The right to
% distribute any portion of psfig/tex for profit or as part of any commercial
% product is specifically reserved for the author(s) of that portion.
%
% *** Feel free to make local modifications of psfig as you wish,
% *** but DO NOT post any changed or modified versions of ``psfig''
% *** directly to the net. Send them to me and I'll try to incorporate
% *** them into future versions. If you want to take the psfig code 
% *** and make a new program (subject to the copyright above), distribute it, 
% *** (and maintain it) that's fine, just don't call it psfig.
%
% Bugs and improvements to trevor@media.mit.edu.
%
% Thanks to Greg Hager (GDH) and Ned Batchelder for their contributions
% to the original version of this project.
%
% Modified by J. Daniel Smith on 9 October 1990 to accept the
% %%BoundingBox: comment with or without a space after the colon.  Stole
% file reading code from Tom Rokicki's EPSF.TEX file (see below).
%
% More modifications by J. Daniel Smith on 29 March 1991 to allow the
% the included PostScript figure to be rotated.  The amount of
% rotation is specified by the "angle=" parameter of the \psfig command.
%
% Modified by Robert Russell on June 25, 1991 to allow users to specify
% .ps filenames which don't yet exist, provided they explicitly provide
% boundingbox information via the \psfig command. Note: This will only work
% if the "file=" parameter follows all four "bb???=" parameters in the
% command. This is due to the order in which psfig interprets these params.
%
%  3 Jul 1991	JDS	check if file already read in once
%  4 Sep 1991	JDS	fixed incorrect computation of rotated
%			bounding box
% 25 Sep 1991	GVR	expanded synopsis of \psfig
% 14 Oct 1991	JDS	\fbox code from LaTeX so \psdraft works with TeX
%			changed \typeout to \ps@typeout
% 17 Oct 1991	JDS	added \psscalefirst and \psrotatefirst
%

% From: gvr@cs.brown.edu (George V. Reilly)
%
% \psdraft	draws an outline box, but doesn't include the figure
%		in the DVI file.  Useful for previewing.
%
% \psfull	includes the figure in the DVI file (default).
%
% \psscalefirst width= or height= specifies the size of the figure
% 		before rotation.
% \psrotatefirst (default) width= or height= specifies the size of the
% 		 figure after rotation.  Asymetric figures will
% 		 appear to shrink.
%
% \psfigurepath#1	sets the path to search for the figure
%
% \psfig
% usage: \psfig{file=, figure=, height=, width=,
%			bbllx=, bblly=, bburx=, bbury=,
%			rheight=, rwidth=, clip=, angle=, silent=}
%
%	"file" is the filename.  If no path name is specified and the
%		file is not found in the current directory,
%		it will be looked for in directory \psfigurepath.
%	"figure" is a synonym for "file".
%	By default, the width and height of the figure are taken from
%		the BoundingBox of the figure.
%	If "width" is specified, the figure is scaled so that it has
%		the specified width.  Its height changes proportionately.
%	If "height" is specified, the figure is scaled so that it has
%		the specified height.  Its width changes proportionately.
%	If both "width" and "height" are specified, the figure is scaled
%		anamorphically.
%	"bbllx", "bblly", "bburx", and "bbury" control the PostScript
%		BoundingBox.  If these four values are specified
%               *before* the "file" option, the PSFIG will not try to
%               open the PostScript file.
%	"rheight" and "rwidth" are the reserved height and width
%		of the figure, i.e., how big TeX actually thinks
%		the figure is.  They default to "width" and "height".
%	The "clip" option ensures that no portion of the figure will
%		appear outside its BoundingBox.  "clip=" is a switch and
%		takes no value, but the `=' must be present.
%	The "angle" option specifies the angle of rotation (degrees, ccw).
%	The "silent" option makes \psfig work silently.
%

% check to see if macros already loaded in (maybe some other file says
% "\input psfig") ...
\ifx\undefined\psfig\else\endinput\fi

%
% from a suggestion by eijkhout@csrd.uiuc.edu to allow
% loading as a style file. Changed to avoid problems
% with amstex per suggestion by jbence@math.ucla.edu

\let\LaTeXAtSign=\@
\let\@=\relax
\edef\psfigRestoreAt{\catcode`\@=\number\catcode`@\relax}
%\edef\psfigRestoreAt{\catcode`@=\number\catcode`@\relax}
\catcode`\@=11\relax
\newwrite\@unused
\def\ps@typeout#1{{\let\protect\string\immediate\write\@unused{#1}}}
\ps@typeout{psfig/tex \PsfigVersion}

%% Here's how you define your figure path.  Should be set up with null
%% default and a user useable definition.

\def\figurepath{./}
\def\psfigurepath#1{\edef\figurepath{#1}}

%
% @psdo control structure -- similar to Latex @for.
% I redefined these with different names so that psfig can
% be used with TeX as well as LaTeX, and so that it will not 
% be vunerable to future changes in LaTeX's internal
% control structure,
%
\def\@nnil{\@nil}
\def\@empty{}
\def\@psdonoop#1\@@#2#3{}
\def\@psdo#1:=#2\do#3{\edef\@psdotmp{#2}\ifx\@psdotmp\@empty \else
    \expandafter\@psdoloop#2,\@nil,\@nil\@@#1{#3}\fi}
\def\@psdoloop#1,#2,#3\@@#4#5{\def#4{#1}\ifx #4\@nnil \else
       #5\def#4{#2}\ifx #4\@nnil \else#5\@ipsdoloop #3\@@#4{#5}\fi\fi}
\def\@ipsdoloop#1,#2\@@#3#4{\def#3{#1}\ifx #3\@nnil 
       \let\@nextwhile=\@psdonoop \else
      #4\relax\let\@nextwhile=\@ipsdoloop\fi\@nextwhile#2\@@#3{#4}}
\def\@tpsdo#1:=#2\do#3{\xdef\@psdotmp{#2}\ifx\@psdotmp\@empty \else
    \@tpsdoloop#2\@nil\@nil\@@#1{#3}\fi}
\def\@tpsdoloop#1#2\@@#3#4{\def#3{#1}\ifx #3\@nnil 
       \let\@nextwhile=\@psdonoop \else
      #4\relax\let\@nextwhile=\@tpsdoloop\fi\@nextwhile#2\@@#3{#4}}
% 
% \fbox is defined in latex.tex; so if \fbox is undefined, assume that
% we are not in LaTeX.
% Perhaps this could be done better???
\ifx\undefined\fbox
% \fbox code from modified slightly from LaTeX
\newdimen\fboxrule
\newdimen\fboxsep
\newdimen\ps@tempdima
\newbox\ps@tempboxa
\fboxsep = 3pt
\fboxrule = .4pt
\long\def\fbox#1{\leavevmode\setbox\ps@tempboxa\hbox{#1}\ps@tempdima\fboxrule
    \advance\ps@tempdima \fboxsep \advance\ps@tempdima \dp\ps@tempboxa
   \hbox{\lower \ps@tempdima\hbox
  {\vbox{\hrule height \fboxrule
          \hbox{\vrule width \fboxrule \hskip\fboxsep
          \vbox{\vskip\fboxsep \box\ps@tempboxa\vskip\fboxsep}\hskip 
                 \fboxsep\vrule width \fboxrule}
                 \hrule height \fboxrule}}}}
\fi
%
%%%%%%%%%%%%%%%%%%%%%%%%%%%%%%%%%%%%%%%%%%%%%%%%%%%%%%%%%%%%%%%%%%%
% file reading stuff from epsf.tex
%   EPSF.TEX macro file:
%   Written by Tomas Rokicki of Radical Eye Software, 29 Mar 1989.
%   Revised by Don Knuth, 3 Jan 1990.
%   Revised by Tomas Rokicki to accept bounding boxes with no
%      space after the colon, 18 Jul 1990.
%   Portions modified/removed for use in PSFIG package by
%      J. Daniel Smith, 9 October 1990.
%
\newread\ps@stream
\newif\ifnot@eof       % continue looking for the bounding box?
\newif\if@noisy        % report what you're making?
\newif\if@atend        % %%BoundingBox: has (at end) specification
\newif\if@psfile       % does this look like a PostScript file?
%
% PostScript files should start with `%!'
%
{\catcode`\%=12\global\gdef\epsf@start{%!}}
\def\epsf@PS{PS}
%
\def\epsf@getbb#1{%
%
%   The first thing we need to do is to open the
%   PostScript file, if possible.
%
\openin\ps@stream=#1
\ifeof\ps@stream\ps@typeout{Error, File #1 not found}\else
%
%   Okay, we got it. Now we'll scan lines until we find one that doesn't
%   start with %. We're looking for the bounding box comment.
%
   {\not@eoftrue \chardef\other=12
    \def\do##1{\catcode`##1=\other}\dospecials \catcode`\ =10
    \loop
       \if@psfile
	  \read\ps@stream to \epsf@fileline
       \else{
	  \obeyspaces
          \read\ps@stream to \epsf@tmp\global\let\epsf@fileline\epsf@tmp}
       \fi
       \ifeof\ps@stream\not@eoffalse\else
%
%   Check the first line for `%!'.  Issue a warning message if its not
%   there, since the file might not be a PostScript file.
%
       \if@psfile\else
       \expandafter\epsf@test\epsf@fileline:. \\%
       \fi
%
%   We check to see if the first character is a % sign;
%   if so, we look further and stop only if the line begins with
%   `%%BoundingBox:' and the `(atend)' specification was not found.
%   That is, the only way to stop is when the end of file is reached,
%   or a `%%BoundingBox: llx lly urx ury' line is found.
%
          \expandafter\epsf@aux\epsf@fileline:. \\%
       \fi
   \ifnot@eof\repeat
   }\closein\ps@stream\fi}%
%
% This tests if the file we are reading looks like a PostScript file.
%
\long\def\epsf@test#1#2#3:#4\\{\def\epsf@testit{#1#2}
			\ifx\epsf@testit\epsf@start\else
\ps@typeout{Warning! File does not start with `\epsf@start'.  It may not be a PostScript file.}
			\fi
			\@psfiletrue} % don't test after 1st line
%
%   We still need to define the tricky \epsf@aux macro. This requires
%   a couple of magic constants for comparison purposes.
%
{\catcode`\%=12\global\let\epsf@percent=%\global\def\epsf@bblit{%BoundingBox}}
%
%
%   So we're ready to check for `%BoundingBox:' and to grab the
%   values if they are found.  We continue searching if `(at end)'
%   was found after the `%BoundingBox:'.
%
\long\def\epsf@aux#1#2:#3\\{\ifx#1\epsf@percent
   \def\epsf@testit{#2}\ifx\epsf@testit\epsf@bblit
	\@atendfalse
        \epsf@atend #3 . \\%
	\if@atend	
	   \if@verbose{
		\ps@typeout{psfig: found `(atend)'; continuing search}
	   }\fi
        \else
        \epsf@grab #3 . . . \\%
        \not@eoffalse
        \global\no@bbfalse
        \fi
   \fi\fi}%
%
%   Here we grab the values and stuff them in the appropriate definitions.
%
\def\epsf@grab #1 #2 #3 #4 #5\\{%
   \global\def\epsf@llx{#1}\ifx\epsf@llx\empty
      \epsf@grab #2 #3 #4 #5 .\\\else
   \global\def\epsf@lly{#2}%
   \global\def\epsf@urx{#3}\global\def\epsf@ury{#4}\fi}%
%
% Determine if the stuff following the %%BoundingBox is `(atend)'
% J. Daniel Smith.  Copied from \epsf@grab above.
%
\def\epsf@atendlit{(atend)} 
\def\epsf@atend #1 #2 #3\\{%
   \def\epsf@tmp{#1}\ifx\epsf@tmp\empty
      \epsf@atend #2 #3 .\\\else
   \ifx\epsf@tmp\epsf@atendlit\@atendtrue\fi\fi}


% End of file reading stuff from epsf.tex
%%%%%%%%%%%%%%%%%%%%%%%%%%%%%%%%%%%%%%%%%%%%%%%%%%%%%%%%%%%%%%%%%%%

%%%%%%%%%%%%%%%%%%%%%%%%%%%%%%%%%%%%%%%%%%%%%%%%%%%%%%%%%%%%%%%%%%%
% trigonometry stuff from "trig.tex"
\chardef\psletter = 11 % won't conflict with \begin{letter} now...
\chardef\other = 12

\newif \ifdebug %%% turn me on to see TeX hard at work ...
\newif\ifc@mpute %%% don't need to compute some values
\c@mputetrue % but assume that we do

\let\then = \relax
\def\r@dian{pt }
\let\r@dians = \r@dian
\let\dimensionless@nit = \r@dian
\let\dimensionless@nits = \dimensionless@nit
\def\internal@nit{sp }
\let\internal@nits = \internal@nit
\newif\ifstillc@nverging
\def \Mess@ge #1{\ifdebug \then \message {#1} \fi}

{ %%% Things that need abnormal catcodes %%%
	\catcode `\@ = \psletter
	\gdef \nodimen {\expandafter \n@dimen \the \dimen}
	\gdef \term #1 #2 #3%
	       {\edef \t@ {\the #1}%%% freeze parameter 1 (count, by value)
		\edef \t@@ {\expandafter \n@dimen \the #2\r@dian}%
				   %%% freeze parameter 2 (dimen, by value)
		\t@rm {\t@} {\t@@} {#3}%
	       }
	\gdef \t@rm #1 #2 #3%
	       {{%
		\count 0 = 0
		\dimen 0 = 1 \dimensionless@nit
		\dimen 2 = #2\relax
		\Mess@ge {Calculating term #1 of \nodimen 2}%
		\loop
		\ifnum	\count 0 < #1
		\then	\advance \count 0 by 1
			\Mess@ge {Iteration \the \count 0 \space}%
			\Multiply \dimen 0 by {\dimen 2}%
			\Mess@ge {After multiplication, term = \nodimen 0}%
			\Divide \dimen 0 by {\count 0}%
			\Mess@ge {After division, term = \nodimen 0}%
		\repeat
		\Mess@ge {Final value for term #1 of 
				\nodimen 2 \space is \nodimen 0}%
		\xdef \Term {#3 = \nodimen 0 \r@dians}%
		\aftergroup \Term
	       }}
	\catcode `\p = \other
	\catcode `\t = \other
	\gdef \n@dimen #1pt{#1} %%% throw away the ``pt''
}

\def \Divide #1by #2{\divide #1 by #2} %%% just a synonym

\def \Multiply #1by #2%%% allows division of a dimen by a dimen
       {{%%% should really freeze parameter 2 (dimen, passed by value)
	\count 0 = #1\relax
	\count 2 = #2\relax
	\count 4 = 65536
	\Mess@ge {Before scaling, count 0 = \the \count 0 \space and
			count 2 = \the \count 2}%
	\ifnum	\count 0 > 32767 %%% do our best to avoid overflow
	\then	\divide \count 0 by 4
		\divide \count 4 by 4
	\else	\ifnum	\count 0 < -32767
		\then	\divide \count 0 by 4
			\divide \count 4 by 4
		\else
		\fi
	\fi
	\ifnum	\count 2 > 32767 %%% while retaining reasonable accuracy
	\then	\divide \count 2 by 4
		\divide \count 4 by 4
	\else	\ifnum	\count 2 < -32767
		\then	\divide \count 2 by 4
			\divide \count 4 by 4
		\else
		\fi
	\fi
	\multiply \count 0 by \count 2
	\divide \count 0 by \count 4
	\xdef \product {#1 = \the \count 0 \internal@nits}%
	\aftergroup \product
       }}

\def\r@duce{\ifdim\dimen0 > 90\r@dian \then   % sin(x+90) = sin(180-x)
		\multiply\dimen0 by -1
		\advance\dimen0 by 180\r@dian
		\r@duce
	    \else \ifdim\dimen0 < -90\r@dian \then  % sin(-x) = sin(360+x)
		\advance\dimen0 by 360\r@dian
		\r@duce
		\fi
	    \fi}

\def\Sine#1%
       {{%
	\dimen 0 = #1 \r@dian
	\r@duce
	\ifdim\dimen0 = -90\r@dian \then
	   \dimen4 = -1\r@dian
	   \c@mputefalse
	\fi
	\ifdim\dimen0 = 90\r@dian \then
	   \dimen4 = 1\r@dian
	   \c@mputefalse
	\fi
	\ifdim\dimen0 = 0\r@dian \then
	   \dimen4 = 0\r@dian
	   \c@mputefalse
	\fi
%
	\ifc@mpute \then
        	% convert degrees to radians
		\divide\dimen0 by 180
		\dimen0=3.141592654\dimen0
%
		\dimen 2 = 3.1415926535897963\r@dian %%% a well-known constant
		\divide\dimen 2 by 2 %%% we only deal with -pi/2 : pi/2
		\Mess@ge {Sin: calculating Sin of \nodimen 0}%
		\count 0 = 1 %%% see power-series expansion for sine
		\dimen 2 = 1 \r@dian %%% ditto
		\dimen 4 = 0 \r@dian %%% ditto
		\loop
			\ifnum	\dimen 2 = 0 %%% then we've done
			\then	\stillc@nvergingfalse 
			\else	\stillc@nvergingtrue
			\fi
			\ifstillc@nverging %%% then calculate next term
			\then	\term {\count 0} {\dimen 0} {\dimen 2}%
				\advance \count 0 by 2
				\count 2 = \count 0
				\divide \count 2 by 2
				\ifodd	\count 2 %%% signs alternate
				\then	\advance \dimen 4 by \dimen 2
				\else	\advance \dimen 4 by -\dimen 2
				\fi
		\repeat
	\fi		
			\xdef \sine {\nodimen 4}%
       }}

% Now the Cosine can be calculated easily by calling \Sine
\def\Cosine#1{\ifx\sine\UnDefined\edef\Savesine{\relax}\else
		             \edef\Savesine{\sine}\fi
	{\dimen0=#1\r@dian\advance\dimen0 by 90\r@dian
	 \Sine{\nodimen 0}
	 \xdef\cosine{\sine}
	 \xdef\sine{\Savesine}}}	      
% end of trig stuff
%%%%%%%%%%%%%%%%%%%%%%%%%%%%%%%%%%%%%%%%%%%%%%%%%%%%%%%%%%%%%%%%%%%%

\def\psdraft{
	\def\@psdraft{0}
	%\ps@typeout{draft level now is \@psdraft \space . }
}
\def\psfull{
	\def\@psdraft{100}
	%\ps@typeout{draft level now is \@psdraft \space . }
}

\psfull

\newif\if@scalefirst
\def\psscalefirst{\@scalefirsttrue}
\def\psrotatefirst{\@scalefirstfalse}
\psrotatefirst

\newif\if@draftbox
\def\psnodraftbox{
	\@draftboxfalse
}
\def\psdraftbox{
	\@draftboxtrue
}
\@draftboxtrue

\newif\if@prologfile
\newif\if@postlogfile
\def\pssilent{
	\@noisyfalse
}
\def\psnoisy{
	\@noisytrue
}
\psnoisy
%%% These are for the option list.
%%% A specification of the form a = b maps to calling \@p@@sa{b}
\newif\if@bbllx
\newif\if@bblly
\newif\if@bburx
\newif\if@bbury
\newif\if@height
\newif\if@width
\newif\if@rheight
\newif\if@rwidth
\newif\if@angle
\newif\if@clip
\newif\if@verbose
\def\@p@@sclip#1{\@cliptrue}


\newif\if@decmpr

%%% GDH 7/26/87 -- changed so that it first looks in the local directory,
%%% then in a specified global directory for the ps file.
%%% RPR 6/25/91 -- changed so that it defaults to user-supplied name if
%%% boundingbox info is specified, assuming graphic will be created by
%%% print time.
%%% TJD 10/19/91 -- added bbfile vs. file distinction, and @decmpr flag

\def\@p@@sfigure#1{\def\@p@sfile{null}\def\@p@sbbfile{null}
	        \openin1=#1.bb
		\ifeof1\closein1
	        	\openin1=\figurepath#1.bb
			\ifeof1\closein1
			        \openin1=#1
				\ifeof1\closein1%
				       \openin1=\figurepath#1
					\ifeof1
					   \ps@typeout{Error, File #1 not found}
						\if@bbllx\if@bblly
				   		\if@bburx\if@bbury
			      				\def\@p@sfile{#1}%
			      				\def\@p@sbbfile{#1}%
							\@decmprfalse
				  	   	\fi\fi\fi\fi
					\else\closein1
				    		\def\@p@sfile{\figurepath#1}%
				    		\def\@p@sbbfile{\figurepath#1}%
						\@decmprfalse
	                       		\fi%
			 	\else\closein1%
					\def\@p@sfile{#1}
					\def\@p@sbbfile{#1}
					\@decmprfalse
			 	\fi
			\else
				\def\@p@sfile{\figurepath#1}
				\def\@p@sbbfile{\figurepath#1.bb}
				\@decmprtrue
			\fi
		\else
			\def\@p@sfile{#1}
			\def\@p@sbbfile{#1.bb}
			\@decmprtrue
		\fi}

\def\@p@@sfile#1{\@p@@sfigure{#1}}

\def\@p@@sbbllx#1{
		%\ps@typeout{bbllx is #1}
		\@bbllxtrue
		\dimen100=#1
		\edef\@p@sbbllx{\number\dimen100}
}
\def\@p@@sbblly#1{
		%\ps@typeout{bblly is #1}
		\@bbllytrue
		\dimen100=#1
		\edef\@p@sbblly{\number\dimen100}
}
\def\@p@@sbburx#1{
		%\ps@typeout{bburx is #1}
		\@bburxtrue
		\dimen100=#1
		\edef\@p@sbburx{\number\dimen100}
}
\def\@p@@sbbury#1{
		%\ps@typeout{bbury is #1}
		\@bburytrue
		\dimen100=#1
		\edef\@p@sbbury{\number\dimen100}
}
\def\@p@@sheight#1{
		\@heighttrue
		\dimen100=#1
   		\edef\@p@sheight{\number\dimen100}
		%\ps@typeout{Height is \@p@sheight}
}
\def\@p@@swidth#1{
		%\ps@typeout{Width is #1}
		\@widthtrue
		\dimen100=#1
		\edef\@p@swidth{\number\dimen100}
}
\def\@p@@srheight#1{
		%\ps@typeout{Reserved height is #1}
		\@rheighttrue
		\dimen100=#1
		\edef\@p@srheight{\number\dimen100}
}
\def\@p@@srwidth#1{
		%\ps@typeout{Reserved width is #1}
		\@rwidthtrue
		\dimen100=#1
		\edef\@p@srwidth{\number\dimen100}
}
\def\@p@@sangle#1{
		%\ps@typeout{Rotation is #1}
		\@angletrue
%		\dimen100=#1
		\edef\@p@sangle{#1} %\number\dimen100}
}
\def\@p@@ssilent#1{ 
		\@verbosefalse
}
\def\@p@@sprolog#1{\@prologfiletrue\def\@prologfileval{#1}}
\def\@p@@spostlog#1{\@postlogfiletrue\def\@postlogfileval{#1}}
\def\@cs@name#1{\csname #1\endcsname}
\def\@setparms#1=#2,{\@cs@name{@p@@s#1}{#2}}
%
% initialize the defaults (size the size of the figure)
%
\def\ps@init@parms{
		\@bbllxfalse \@bbllyfalse
		\@bburxfalse \@bburyfalse
		\@heightfalse \@widthfalse
		\@rheightfalse \@rwidthfalse
		\def\@p@sbbllx{}\def\@p@sbblly{}
		\def\@p@sbburx{}\def\@p@sbbury{}
		\def\@p@sheight{}\def\@p@swidth{}
		\def\@p@srheight{}\def\@p@srwidth{}
		\def\@p@sangle{0}
		\def\@p@sfile{} \def\@p@sbbfile{}
		\def\@p@scost{10}
		\def\@sc{}
		\@prologfilefalse
		\@postlogfilefalse
		\@clipfalse
		\if@noisy
			\@verbosetrue
		\else
			\@verbosefalse
		\fi
}
%
% Go through the options setting things up.
%
\def\parse@ps@parms#1{
	 	\@psdo\@psfiga:=#1\do
		   {\expandafter\@setparms\@psfiga,}}
%
% Compute bb height and width
%
\newif\ifno@bb
\def\bb@missing{
	\if@verbose{
		\ps@typeout{psfig: searching \@p@sbbfile \space  for bounding box}
	}\fi
	\no@bbtrue
	\epsf@getbb{\@p@sbbfile}
        \ifno@bb \else \bb@cull\epsf@llx\epsf@lly\epsf@urx\epsf@ury\fi
}	
\def\bb@cull#1#2#3#4{
	\dimen100=#1 bp\edef\@p@sbbllx{\number\dimen100}
	\dimen100=#2 bp\edef\@p@sbblly{\number\dimen100}
	\dimen100=#3 bp\edef\@p@sbburx{\number\dimen100}
	\dimen100=#4 bp\edef\@p@sbbury{\number\dimen100}
	\no@bbfalse
}
% rotate point (#1,#2) about (0,0).
% The sine and cosine of the angle are already stored in \sine and
% \cosine.  The result is placed in (\p@intvaluex, \p@intvaluey).
\newdimen\p@intvaluex
\newdimen\p@intvaluey
\def\rotate@#1#2{{\dimen0=#1 sp\dimen1=#2 sp
%            	calculate x' = x \cos\theta - y \sin\theta
		  \global\p@intvaluex=\cosine\dimen0
		  \dimen3=\sine\dimen1
		  \global\advance\p@intvaluex by -\dimen3
% 		calculate y' = x \sin\theta + y \cos\theta
		  \global\p@intvaluey=\sine\dimen0
		  \dimen3=\cosine\dimen1
		  \global\advance\p@intvaluey by \dimen3
		  }}
\def\compute@bb{
		\no@bbfalse
		\if@bbllx \else \no@bbtrue \fi
		\if@bblly \else \no@bbtrue \fi
		\if@bburx \else \no@bbtrue \fi
		\if@bbury \else \no@bbtrue \fi
		\ifno@bb \bb@missing \fi
		\ifno@bb \ps@typeout{FATAL ERROR: no bb supplied or found}
			\no-bb-error
		\fi
		%
%\ps@typeout{BB: \@p@sbbllx, \@p@sbblly, \@p@sbburx, \@p@sbbury} 
%
% store height/width of original (unrotated) bounding box
		\count203=\@p@sbburx
		\count204=\@p@sbbury
		\advance\count203 by -\@p@sbbllx
		\advance\count204 by -\@p@sbblly
		\edef\ps@bbw{\number\count203}
		\edef\ps@bbh{\number\count204}
		%\ps@typeout{ psbbh = \ps@bbh, psbbw = \ps@bbw }
		\if@angle 
			\Sine{\@p@sangle}\Cosine{\@p@sangle}
	        	{\dimen100=\maxdimen\xdef\r@p@sbbllx{\number\dimen100}
					    \xdef\r@p@sbblly{\number\dimen100}
			                    \xdef\r@p@sbburx{-\number\dimen100}
					    \xdef\r@p@sbbury{-\number\dimen100}}
%
% Need to rotate all four points and take the X-Y extremes of the new
% points as the new bounding box.
                        \def\minmaxtest{
			   \ifnum\number\p@intvaluex<\r@p@sbbllx
			      \xdef\r@p@sbbllx{\number\p@intvaluex}\fi
			   \ifnum\number\p@intvaluex>\r@p@sbburx
			      \xdef\r@p@sbburx{\number\p@intvaluex}\fi
			   \ifnum\number\p@intvaluey<\r@p@sbblly
			      \xdef\r@p@sbblly{\number\p@intvaluey}\fi
			   \ifnum\number\p@intvaluey>\r@p@sbbury
			      \xdef\r@p@sbbury{\number\p@intvaluey}\fi
			   }
%			lower left
			\rotate@{\@p@sbbllx}{\@p@sbblly}
			\minmaxtest
%			upper left
			\rotate@{\@p@sbbllx}{\@p@sbbury}
			\minmaxtest
%			lower right
			\rotate@{\@p@sbburx}{\@p@sbblly}
			\minmaxtest
%			upper right
			\rotate@{\@p@sbburx}{\@p@sbbury}
			\minmaxtest
			\edef\@p@sbbllx{\r@p@sbbllx}\edef\@p@sbblly{\r@p@sbblly}
			\edef\@p@sbburx{\r@p@sbburx}\edef\@p@sbbury{\r@p@sbbury}
%\ps@typeout{rotated BB: \r@p@sbbllx, \r@p@sbblly, \r@p@sbburx, \r@p@sbbury}
		\fi
		\count203=\@p@sbburx
		\count204=\@p@sbbury
		\advance\count203 by -\@p@sbbllx
		\advance\count204 by -\@p@sbblly
		\edef\@bbw{\number\count203}
		\edef\@bbh{\number\count204}
		%\ps@typeout{ bbh = \@bbh, bbw = \@bbw }
}
%
% \in@hundreds performs #1 * (#2 / #3) correct to the hundreds,
%	then leaves the result in @result
%
\def\in@hundreds#1#2#3{\count240=#2 \count241=#3
		     \count100=\count240	% 100 is first digit #2/#3
		     \divide\count100 by \count241
		     \count101=\count100
		     \multiply\count101 by \count241
		     \advance\count240 by -\count101
		     \multiply\count240 by 10
		     \count101=\count240	%101 is second digit of #2/#3
		     \divide\count101 by \count241
		     \count102=\count101
		     \multiply\count102 by \count241
		     \advance\count240 by -\count102
		     \multiply\count240 by 10
		     \count102=\count240	% 102 is the third digit
		     \divide\count102 by \count241
		     \count200=#1\count205=0
		     \count201=\count200
			\multiply\count201 by \count100
		 	\advance\count205 by \count201
		     \count201=\count200
			\divide\count201 by 10
			\multiply\count201 by \count101
			\advance\count205 by \count201
			%
		     \count201=\count200
			\divide\count201 by 100
			\multiply\count201 by \count102
			\advance\count205 by \count201
			%
		     \edef\@result{\number\count205}
}
\def\compute@wfromh{
		% computing : width = height * (bbw / bbh)
		\in@hundreds{\@p@sheight}{\@bbw}{\@bbh}
		%\ps@typeout{ \@p@sheight * \@bbw / \@bbh, = \@result }
		\edef\@p@swidth{\@result}
		%\ps@typeout{w from h: width is \@p@swidth}
}
\def\compute@hfromw{
		% computing : height = width * (bbh / bbw)
	        \in@hundreds{\@p@swidth}{\@bbh}{\@bbw}
		%\ps@typeout{ \@p@swidth * \@bbh / \@bbw = \@result }
		\edef\@p@sheight{\@result}
		%\ps@typeout{h from w : height is \@p@sheight}
}
\def\compute@handw{
		\if@height 
			\if@width
			\else
				\compute@wfromh
			\fi
		\else 
			\if@width
				\compute@hfromw
			\else
				\edef\@p@sheight{\@bbh}
				\edef\@p@swidth{\@bbw}
			\fi
		\fi
}
\def\compute@resv{
		\if@rheight \else \edef\@p@srheight{\@p@sheight} \fi
		\if@rwidth \else \edef\@p@srwidth{\@p@swidth} \fi
		%\ps@typeout{rheight = \@p@srheight, rwidth = \@p@srwidth}
}
%		
% Compute any missing values
\def\compute@sizes{
	\compute@bb
	\if@scalefirst\if@angle
% at this point the bounding box has been adjsuted correctly for
% rotation.  PSFIG does all of its scaling using \@bbh and \@bbw.  If
% a width= or height= was specified along with \psscalefirst, then the
% width=/height= value needs to be adjusted to match the new (rotated)
% bounding box size (specifed in \@bbw and \@bbh).
%    \ps@bbw       width=
%    -------  =  ---------- 
%    \@bbw       new width=
% so `new width=' = (width= * \@bbw) / \ps@bbw; where \ps@bbw is the
% width of the original (unrotated) bounding box.
	\if@width
	   \in@hundreds{\@p@swidth}{\@bbw}{\ps@bbw}
	   \edef\@p@swidth{\@result}
	\fi
	\if@height
	   \in@hundreds{\@p@sheight}{\@bbh}{\ps@bbh}
	   \edef\@p@sheight{\@result}
	\fi
	\fi\fi
	\compute@handw
	\compute@resv}

%
% \psfig
% usage : \psfig{file=, height=, width=, bbllx=, bblly=, bburx=, bbury=,
%			rheight=, rwidth=, clip=}
%
% "clip=" is a switch and takes no value, but the `=' must be present.
\def\psfig#1{\vbox {
	% do a zero width hard space so that a single
	% \psfig in a centering enviornment will behave nicely
	%{\setbox0=\hbox{\ }\ \hskip-\wd0}
	%
	\ps@init@parms
	\parse@ps@parms{#1}
	\compute@sizes
	%
	\ifnum\@p@scost<\@psdraft{
		%
		\special{ps::[begin] 	\@p@swidth \space \@p@sheight \space
				\@p@sbbllx \space \@p@sbblly \space
				\@p@sbburx \space \@p@sbbury \space
				startTexFig \space }
		\if@angle
			\special {ps:: \@p@sangle \space rotate \space} 
		\fi
		\if@clip{
			\if@verbose{
				\ps@typeout{(clip)}
			}\fi
			\special{ps:: doclip \space }
		}\fi
		\if@prologfile
		    \special{ps: plotfile \@prologfileval \space } \fi
		\if@decmpr{
			\if@verbose{
				\ps@typeout{psfig: including \@p@sfile.Z \space }
			}\fi
			\special{ps: plotfile "`zcat \@p@sfile.Z" \space }
		}\else{
			\if@verbose{
				\ps@typeout{psfig: including \@p@sfile \space }
			}\fi
			\special{ps: plotfile \@p@sfile \space }
		}\fi
		\if@postlogfile
		    \special{ps: plotfile \@postlogfileval \space } \fi
		\special{ps::[end] endTexFig \space }
		% Create the vbox to reserve the space for the figure.
		\vbox to \@p@srheight sp{
		% 1/92 TJD Changed from "true sp" to "sp" for magnification.
			\hbox to \@p@srwidth sp{
				\hss
			}
		\vss
		}
	}\else{
		% draft figure, just reserve the space and print the
		% path name.
		\if@draftbox{		
			% Verbose draft: print file name in box
			\hbox{\frame{\vbox to \@p@srheight sp{
			\vss
			\hbox to \@p@srwidth sp{ \hss \@p@sfile \hss }
			\vss
			}}}
		}\else{
			% Non-verbose draft
			\vbox to \@p@srheight sp{
			\vss
			\hbox to \@p@srwidth sp{\hss}
			\vss
			}
		}\fi	



	}\fi
}}
\psfigRestoreAt
\let\@=\LaTeXAtSign
}

\oddsidemargin=1pt
\evensidemargin=1pt

\newcommand{\loss}[2]{\ell_{#1}^{#2}} %loss of expert i at time t
\newcommand{\TEloss}[1]{L_{#1}}	%total loss of expert i
\newcommand{\BEloss}{L_{\min}}	%total loss of the best expert
\newcommand{\TAloss}{L_A}	%total loss of algorithm
\newcommand{\weight}[2]{W_{#1}^{#2}} % weight assigned to expert
					  % i at time t
\newcommand{\Nweight}[2]{V_{#1}^{#2}}	%the normalized weight
\newcommand{\dweight}[2]{w^{#2}(#1)} % initial density measure
\newcommand{\btheta}{\hat{\theta}}

\newcommand{\bt}{$\bullet$~}

\textwidth=6.5in
\topmargin=-20pt
\headheight=12pt
\headsep=12pt
\textheight=8.5in

\def\topfraction{0.99}
\def\floatpagefraction{0.99}
\def\textfraction{0.1} %squeeze lines onto page even if 90% figures!

\pagestyle{empty}

\begin{document}
\newcommand{\mRule}{\rule{6.35in}{0.01in}}

% Title Slide %%%%%%%%%%%%%%%%%%%%%%%%%%%%%%%%%%%%%%%%%%%%%%%%%%%%%%

\begin{LARGE}
\begin{center}
~\newline \newline \newline \newline 
\begin{Huge}
On-line learning of individual sequences
\end{Huge}
\newline \newline \newline \newline 
Yoav Freund, AT\&T Research
\end{center}
\vspace{1in}
A review talk, some results from collaborations with:\newline
\newline
Peter Auer \newline\newline
Nicol\'{o} Cesa-Bianchi \newline\newline
David Haussler \newline\newline
David Helmbold \newline\newline
Rob Schapire \newline\newline
Manfred Warmuth
~\newline\newline\newline
Papers and transparencies in:\newline
http://www.research.att.com/orgs/ssr/people/yoav
\end{LARGE}

\pagebreak

\begin{flushleft}
\begin{LARGE}

% The general problem  %%%%%%%%%%%%%%%%%%%%%%%%%%%%%%%%%%%%%%%%%%%%%%%%%%%%%%

\begin{center}
\underline{A simple prediction problem}
\end{center}

\begin{itemize}
\item On-line prediction of a binary sequence.
\item $N$ experts provide predictions of the sequence.
\item {\it Assumption:} one of the experts makes no mistakes.
\item {\it Goal:} Predict to make the minimal number of mistakes.
\end{itemize}

{\bf Example:}
\[\begin{array}{lccccc}
Sequence: &0,&1,&1,&0,&\ldots \\
\\
expert_1: &1,&1,&0,&1,&\ldots \\
expert_2: &0,&1,&1,&1,&\ldots \\
expert_3: &0,&1,&1,&0,&\ldots \\
expert_4: &0,&1,&1,&1,&\ldots \\
\end{array} \]

\pagebreak

% %%%%%%%%%%%%%%%%%%%%%%%%%%%%%%%%%%%%%%%%%%%%%%%%%%%%%%%%
\begin{center}
\underline{Solution of simple problem}
\end{center}

\begin{itemize}
\item
Predict according to the majority of experts that have made no mistake.
\item
At each mistake the pool of active experts is {\em halved}.
\end{itemize}
$\Rightarrow$ At most $\log_2 N$ mistakes can be made.
~\newline\newline
~\newline\newline
\underline{Analysis characteristics:}
\begin{itemize}
\item
{\em Online} (iterative) setup.
\item
{\em No} probabilistic assumptions.
\item
Bounds hold for {\em worst-case} sequences of experts, predictions and outcomes.
\end{itemize}
~\newline
Can we generalize to when no expert is perfect?

\pagebreak
% Preview Slide %
% %%%%%%%%%%%%%%%%%%%%%%%%%%%%%%%%%%%%%%%%%%%%%%%%%%%%%%%%
\begin{center}
\underline{Preview}
\end{center}

\begin{enumerate}
\item A simple and general (but not optimal) on-line prediction
      algorithm.
\item An even more general (but non-constructive) meta-algorithm.
\item Detailed results for specific cases.
\item Continuous sets of experts.
%\item Multiple-armed bandits problem.
%\item Efficient algorithms for exponentially many experts.
\end{enumerate}

\pagebreak
% %%%%%%%%%%%%%%%%%%%%%%%%%%%%%%%%%%%%%%%%%%%%%%%%%%%%%%%%
\begin{center}
\underline{Predicting when no expert is perfect}
\end{center}

\underline{Prediction game: algorithm vs. adversary}
\newline\newline
for t=1,2,\ldots,T
\begin{enumerate}
\item
{\bf Adversary:} Each expert suggest an action. \newline
	   A loss $\loss{i}{t} \in [0,1]$ is associated with each action.
\item
{\bf Algorithm:} Chooses an expert $i_t$ (can randomize).
\item
All losses $\loss{1}{t},\ldots,\loss{N}{t}$ are revealed to player.
\end{enumerate}
\pagebreak
% %%%%%%%%%%%%%%%%%%%%%%%%%%%%%%%%%%%%%%%%%%%%%%%%%%%%%%%%
\begin{center}
\underline{Quantities of interest}
\end{center}
Total loss of expert $i$: $$\TEloss{i} \doteq \sum_{t=1}^T \loss{i}{t}$$
Total loss of best expert: $$\BEloss \doteq \min_{i=1,\ldots,N}
\TEloss{i}$$
Expected total loss of algorithm:
$$\TAloss \doteq E_{i_1,\ldots,i_t} \left[ \sum_{t=1}^T \loss{i_t}{t} \right]$$
\newline
{\em Goal:} Minimize $\TAloss - \BEloss$ in the worst case.

\pagebreak
	
% %%%%%%%%%%%%%%%%%%%%%%%%%%%%%%%%%%%%%%%%%%%%%%%%%%%%%%%%
\begin{center}
\underline{Algorithm Hedge (Weighted Majority)}
\end{center}
[Littlestone, Warmuth 89],[Freund, Schapire 95]\newline

``Weight'' $\weight{i}{t}>0$ associated with expert $i$ at
time $t$.
\newline\newline
One parameter: $\eta>0$.
\newline\newline
{\bf Initial weights:} $\forall i \in \{1 \ldots N\}\;\;
\weight{i}{1}=1/N$
\begin{enumerate}
\item
Prob. of choosing expert $i$ on iteration $t$:
\[ \weight{i}{t} \over \sum_{j=1}^N \weight{j}{t}\]
\item
{Weights update after iteration $t$:} 
\[ \weight{i}{t+1} = \weight{i}{t}\; e^{-\eta \loss{i}{t}} \]
\end{enumerate}
\pagebreak
% %%%%%%%%%%%%%%%%%%%%%%%%%%%%%%%%%%%%%%%%%%%%%%%%%%%%%%%%
\begin{center}
\underline{Theorem regarding the performance of Hedge}
\end{center}

For {\em any} sequence of predictions and outcomes of length $T$
\[
\TAloss \leq {\eta \BEloss + \ln N \over 1-e^{-\eta}}
\]
~\newline\newline
Setting $\eta$ to minimize bound:
\begin{itemize}
\item
Decreasing $\eta$ decreases numerator.
\item
If $\eta \to 0$ denominator becomes zero.
\end{itemize}
~\newline\newline
{\bf Corollary}
For any $T$
\[
\mbox{If  } \eta = \ln \left(1+\sqrt{2 \ln N \over T} \right)
\]
then
\[
\TAloss - \BEloss \leq \sqrt{2T \ln N} +\ln N~.
\]
In other words:
\[
{\TAloss \over T} \leq  {\BEloss \over T} + \sqrt{2\ln N \over T}
+{\ln N \over T}~.
\]
\pagebreak

% %%%%%%%%%%%%%%%%%%%%%%%%%%%%%%%%%%%%%%%%%%%%%%%%%%%%%%%%
\begin{center}
\underline{Proof sketch, Main argument}
\end{center}
~\newline
{\bf Lemma I:}
If expected loss is large, final total weight is small:
\[
\ln \sum_i \weight{i}{T+1} 
 \leq - (1-e^{-\eta})
	\sum_{t=1}^T E_{i_t} [\loss{i_t}{t}]
\]
~\newline
{\bf Lemma II:}
If the best expert is good, the final weight cannot be too small:

\[
\sum_i \weight{i}{T+1} 
 \geq {1 \over N} \exp (-\eta \BEloss)
\]
~\newline
Combining the bounds we get
\[
- (1-e^{-\eta})
	\sum_{t=1}^T E_{i_t} [\loss{i_t}{t}]
\geq \ln \left({1 \over N} \exp (-\eta \BEloss) \right)
\]
\[
\sum_{t=1}^T E_{i_t} [\loss{i_t}{t}]
\leq
{ \ln N + \eta \BEloss \over 1-e^{-\eta}}
\]

\pagebreak

% %%%%%%%%%%%%%%%%%%%%%%%%%%%%%%%%%%%%%%%%%%%%%%%%%%%%%%%%
\begin{center}
\underline{Proof of lemma II}
\end{center}

For {\em any} $j \in \{1\ldots N\}$:
\[
\sum_i \weight{i}{T+1} \geq \weight{j}{T+1} 
= {1 \over N} \exp (-\eta \TEloss{j})
\]

\pagebreak

% %%%%%%%%%%%%%%%%%%%%%%%%%%%%%%%%%%%%%%%%%%%%%%%%%%%%%%%%
\begin{center}
\underline{Proof of lemma I}
\end{center}

\begin{eqnarray*}
\sum_i \weight{i}{t+1} 
&=& \sum_i \weight{i}{t} \;\; \exp (-\eta \loss{i}{t}) \\
&\leq& \sum_i \weight{i}{t} \left(1-(1-e^{-\eta})\loss{i}{t}
\right) \\
&=&
\sum_i \weight{i}{t} 
-
(1-e^{-\eta}) \sum_i \weight{i}{t} \loss{i}{t}
\end{eqnarray*}
As
\[
E_{i_t} [\loss{i_t}{t}] = \sum_i 
{\weight{i}{t} \over \sum_j \weight{j}{t}} \loss{i}{t}
\]
We get
\[
{\sum_i \weight{i}{t+1} \over \sum_i \weight{i}{t} }
\leq 1 - (1-e^{-\eta}) E_{i_t} [\loss{i_t}{t}]~.
\]
~\newline
Taking logs and Combining for $t=1\ldots T$:
\begin{eqnarray*}
\ln {\sum_i \weight{i}{T+1} \over \sum_i \weight{i}{1}}
&\leq & \sum_{t=1}^T 
     \ln \left(1-(1-e^{-\eta}) E_{i_t} [\loss{i_t}{t}] \right) \\
&\leq &
     - (1-e^{-\eta})
	\sum_{t=1}^T E_{i_t} [\loss{i_t}{t}]
\end{eqnarray*}
But $\sum_i \weight{i}{1} = N {1 \over N} = 1$.\newline

\pagebreak

% %%%%%%%%%%%%%%%%%%%%%%%%%%%%%%%%%%%%%%%%%%%%%%%%%%%%%%%%
\begin{center}
\underline{What can be done beyond this?}
\end{center}
\begin{itemize}
\item
Unbounded loss per trial.
\item
Better bounds for special losses
\begin{itemize}
\item
Define loss as a function of {\em prediction} and {\em outcome}.
\item
Combine predictions of experts instead of randomly 
choosing a single expert.
\end{itemize}
\item
Better bounds for special model classes
\begin{itemize}
\item
Models parameterized by continuous parameters.
\item
Neighborhood structure of models.
\end{itemize}
%\item
%Efficiency considerations.
%\item 
%restricted feedback
\end{itemize}

\pagebreak

% %%%%%%%%%%%%%%%%%%%%%%%%%%%%%%%%%%%%%%%%%%%%%%%%%%%%%%%%
\begin{center}
\underline{Some useful loss functions}
\end{center}

Outcomes: binary $x^1,x^2,\ldots$ \newline
Predictions: $p^1,p^2,\ldots$  $p^t \in [0,1]$

\begin{itemize}
\item
{\bf Absolute loss (Prediction error)}
\[ \loss{}{t} = | x^t -p^t |\]
Probability of making a mistake if predicting 0 or 1 
using a biased coin\newline
If $P[x^t=1]=q$, then the optimal prediction is 
\[
p^t = \cases{1,&if $q > 1/2$ \cr 
             0,& otherwise}~.
\]
\item
{\bf Log loss (Entropy loss)} 
\[ \loss{}{t} = -x^t \ln p^t -(1-x^t) \ln (1-p^t) \]
Cumulative log loss $=$ coding length $\pm 1$ \newline
If $P[x^t=1]=q$, optimal prediction $p^t=q$.
\item
{\bf Square loss (Breier Loss)}
\[ \loss{}{t} = (x^t-p^t)^2 \]
If $P[x^t=1]=q$, optimal prediction $p^t=q$.\newline
Loss is bounded.\newline
Example: predicting forehand/backhand in ping-pong.
\end{itemize}
\pagebreak

% %%%%%%%%%%%%%%%%%%%%%%%%%%%%%%%%%%%%%%%%%%%%%%%%%%%%%%%%
\begin{center}
\underline{Vovk's general result}
[Vovk 89,95]
\end{center}

{\bf A game, between Nature and a Learner:}\newline
\newline
For $t=1,2,\ldots$:
\begin{enumerate}
\item
Each expert $i \in \{1 \ldots N \}$ makes a prediction 
$\gamma_i^t \in \Gamma$.
\item
The learner, after observing $\langle \gamma_1^t \ldots \gamma_N^T
\rangle$, makes its own prediction $\gamma^t$.
\item
Nature chooses an outcome $\omega^t \in \Omega$.
\item
Each expert incurs loss $\loss{i}{t} = \lambda(\omega^t,\gamma_i^t)$.
The learner incurs loss $\loss{}{t} = \lambda(\omega^t,\gamma^t)$.
\end{enumerate}

{\em Goal:} guarantee that 
\[ \TAloss \leq a \BEloss + c \ln N \]
for any sequence and for the smallest possible pairs $(a,c) \in [0,\infty)^2$.
\pagebreak

% %%%%%%%%%%%%%%%%%%%%%%%%%%%%%%%%%%%%%%%%%%%%%%%%%%%%%%%%
\begin{center}
\underline{The loss function should be well-behaved}
\end{center}
There should be a topology on the set $\Gamma$ such that
\begin{itemize}
\item
$\Gamma$ is compact.
\item
$\forall \omega \in \Omega$, the function $\gamma \to
\lambda(\omega,\gamma)$ is continuous.
\item
There is a universally reasonable prediction
$\exists \gamma \in \Gamma$, $\forall \omega \in \Omega$,
$\lambda(\omega,\gamma) < \infty$.
\item
There is no universally optimal prediction
$\neg \exists \gamma \in \Gamma$, $\forall \omega \in \Omega$,
$\lambda(\omega,\gamma) = 0$.
\end{itemize}
\pagebreak

% %%%%%%%%%%%%%%%%%%%%%%%%%%%%%%%%%%%%%%%%%%%%%%%%%%%%%%%%
\begin{center}
\underline{The set of achievable bounds}
\end{center}
The pair $(a,c)$ is {\em achievable} if there exists 
{\em some} prediction algorithm
such that for {\em any} $N>0$, {\em any} set of $N$ prediction
sequences and {\em any} sequence of outcomes
\[
\TAloss \leq a \BEloss + c \ln N
\]

\withpsfigs{\centerline{\psfig{figure=achievable1.ps,width=7in}}}

\pagebreak

% %%%%%%%%%%%%%%%%%%%%%%%%%%%%%%%%%%%%%%%%%%%%%%%%%%%%%%%%
\begin{center}
\underline{Vovk's meta-algorithm}
\end{center}

~\newline
Fix an achievable pair $(a,c)$.\newline
~\newline
Set $\eta=a/c$.
~\newline

\begin{enumerate}
\item
Define
\[
	\weight{i}{t} = {1 \over N} \; e^{-\eta \TEloss{i}^{t}}
\]
\item
Choose $\gamma_t$ so that, for all $\omega^t \in \Omega$:
\[
\lambda(\omega^t,\gamma^t) - c \ln \sum_i \weight{i}{t} 
\leq
- c \ln \left( \sum_i 
      \weight{i}{t}e^{-\eta \lambda(\omega^t,\gamma_i^t)}
        \right)
\]
\end{enumerate}

If choice of $\gamma^t$ always exists, then the total loss satisfies:
\[
\sum_t \lambda(\omega^t,\gamma^t)
\leq
- c \ln \sum_i \weight{i}{T+1}
\leq
a \BEloss + c \ln N
\]
~\newline
Vovk's result: {\em yes!} a choice for $\gamma_t$ always exists!
\pagebreak

% %%%%%%%%%%%%%%%%%%%%%%%%%%%%%%%%%%%%%%%%%%%%%%%%%%%%%%%%
\begin{center}
\underline{Vovk's algorithm is optimal}
\end{center}
{\bf Theorem (Vovk 95)}\newline
The pair $(a,c)$ is achievable (by some algorithm) 
if and only if it is achieved by Vovk's algorithm.
\newline
The separation curve is
\[ \left\{ \left. \left( a(\eta),{a(\eta) \over \eta} \right) \right| 
           \eta \in [0,\infty] \right\} \]

\withpsfigs{\centerline{\psfig{figure=achievable2.ps,width=7in}}}

\pagebreak
% %%%%%%%%%%%%%%%%%%%%%%%%%%%%%%%%%%%%%%%%%%%%%%%%%%%%%%%%
\begin{center}
\underline{Special case: $a(\eta) = 1$ for $\eta<\infty$}
\end{center}

Vovk's condition:
Choose $\gamma$ so that, for all $\omega \in \Omega$:
\[
\lambda(\omega,\gamma^t) - c \ln \sum_i \weight{i}{t} 
\leq
- c \ln \left( \sum_i 
      \weight{i}{t}e^{-\eta \lambda(\omega,\gamma_i^t)}
        \right)
\]
Can be re-written as:
\[
e^{-{1 \over c} \lambda(\omega,\gamma)}
\geq
\sum_i 
\left({\weight{i}{} \over \sum_j \weight{j}{}}\right)
e^{-\eta \lambda(\omega,\gamma_i)}
\]
\begin{itemize}
\item
$a(\eta)=1$ $\Leftrightarrow$ ${1 \over c} = \eta$.
\item
{\bf Assumption:}
fix $\lambda(\omega,\gamma_i)$ for all but $i \not\in \{j,k\}$
then increasing $\lambda(\omega,\gamma_j)$ decreases
$\lambda(\omega,\gamma_k)$. 
\item
We get a convexity condition on the image of $\Gamma$ under the
function
\[F(\gamma) = 
\left\langle 
   e^{-\eta \lambda(\omega,\gamma)} 
\right\rangle_{\omega \in \Omega}
\]
\end{itemize}

\noindent {\bf Example:} Suppose $\Omega=\{0,1\}$, $\Gamma = [0,1]$.
then
\[
F(\gamma) = \left\langle
  e^{-\eta \lambda(0,\gamma)},e^{-\eta \lambda(1,\gamma)}
\right\rangle
\]
\withpsfigs{\centerline{\psfig{figure=convex.ps,height=2.5in}}}

\pagebreak

% %%%%%%%%%%%%%%%%%%%%%%%%%%%%%%%%%%%%%%%%%%%%%%%%%%%%%%%%
\begin{center}
\underline{Negating Vovk's condition implies a ``bad'' distribution}
\end{center}
Vovk's condition:
\begin{eqnarray*}
& \forall & N,\\
& \forall & \langle (\gamma_1,\weight{1}{}),\ldots,(\gamma_N,\weight{N}{}) \rangle,\\
& \exists & \gamma,\; \forall \omega: \\
&&e^{-{1 \over c} \lambda(\omega,\gamma)}
\geq
\sum_i 
\left({\weight{i}{} \over \sum_j \weight{j}{}}\right)
e^{-\eta \lambda(\omega,\gamma_i)}
\end{eqnarray*}
Negation is
\begin{eqnarray*}
& \exists & N,\\ 
& \exists &
\langle (\gamma_1,\weight{1}{}),\ldots,(\gamma_N,\weight{N}{}) \rangle,\\
& \forall & \gamma,\; \exists \omega: \\
&& e^{-{1 \over c} \lambda(\omega,\gamma)}
<
\sum_i 
\left({\weight{i}{} \over \sum_j \weight{j}{}}\right)
e^{-\eta \lambda(\omega,\gamma_i)}
\end{eqnarray*}
Defines a ``BAD'' distribution over $\Gamma$ (prediction space).\newline
The distribution has a finite support.

\pagebreak
% %%%%%%%%%%%%%%%%%%%%%%%%%%%%%%%%%%%%%%%%%%%%%%%%%%%%%%%%
\begin{center}
\underline{Adversarial construction for Vovk's algorithm}
\end{center}

\begin{itemize}
\item
{\bf Experts}: Each expert predicts IID according to the ``BAD''
distribution over $\Gamma$.
\item
{\bf Sequence:}
Choose an outcome $\omega$ which satisfies the bad inequality {\em
and}:
\begin{itemize}
\item
If $\exists \omega$ such that some expert suffers non-zero loss then use
it.
\item
Else choose $\omega$ that maximizes loss of prediction algorithm.
\end{itemize}
\end{itemize}
~\newline
~\newline
Vovk's homepage: HTTP://casbs.stanford.edu/\~{}vovk/
\pagebreak
% %%%%%%%%%%%%%%%%%%%%%%%%%%%%%%%%%%%%%%%%%%%%%%%%%%%%%%%%
\begin{center}
\underline{Explicit analysis for special cases}
\end{center}

Vovk's meta-algorithm is not constructive:\newline
\newline
Given $\lambda(\omega,\gamma)$ find:
\begin{itemize}
\item
Set of achievable pairs. $\{(a,c)\}$ \newline
In particular, is there an achievable $(1,c)$, $c<\infty$ ?
\item
How to calculate a good prediction $\gamma^t$?\newline
(easy when $|\Omega| < \infty$).
\end{itemize}
\pagebreak

% %%%%%%%%%%%%%%%%%%%%%%%%%%%%%%%%%%%%%%%%%%%%%%%%%%%%%%%%
\begin{center}
\underline{Vovk's algorithm for binary outcomes, log loss}
\end{center}
\[ \loss{}{t} = -x^t \ln p_i^t -(1-x^t) \ln (1-p_i^t) \]

~
\begin{itemize}
\item
The pair $(1,1)$ is achieved by $\eta=1$.
\item
Prediction $=$ weighted average:
\[
	{\sum_i \weight{i}{t} \gamma_i^t
	\over
	 \sum_i \weight{i}{t}}
\]
\item
The update rule is:
\begin{eqnarray*}
\weight{i}{t+1} &=& \weight{i}{t} \; 
                  \exp \left(x^t \ln p_i^t +(1-x^t) \ln (1-p_i^t)\right)\\
&=& \weight{i}{t} (p_i^t)^{x^t} (1-p_i^t)^{1-x^t}
\end{eqnarray*}
\end{itemize}
This algorithm is {\em identical} to the Bayes prediction algorithm with 
a uniform prior.
\newline\newline
{\bf Theorem:} For {\em any} sequence the cumulative log loss of the
algorithm is larger by at most $\ln N$ from the loss of the best
expert.
\newline\newline
This result is well known in Information theory as a ``Universal
coding'' folk-theorem.
\pagebreak

% %%%%%%%%%%%%%%%%%%%%%%%%%%%%%%%%%%%%%%%%%%%%%%%%%%%%%%%%
\begin{center}
\underline{Vovk's algorithm for Binary outcomes, square loss}
\end{center}
\[ \loss{}{t} = (x^t - p_i^t)^2 \]

~
\begin{itemize}
\item
The pair $(1,1/2)$ is achieved by $\eta=2$.
\item
{\bf Weight Update rule:}
\begin{eqnarray*}
\weight{i}{t+1} &=& \weight{i}{t} \; 
                  \exp \left(-2 (x^t - p_i^t)^2 \right)
\end{eqnarray*}
\item
{\bf Prediction rule:} any choice from the range
\[
1-\sqrt{ -{1 \over 2} \ln \sum_i \Nweight{i}{t} e^{-2(1-p^t_i)^2} }
\leq
p^t
\leq
\sqrt{ -{1 \over 2} \ln \sum_i \Nweight{i}{t} e^{-2(p^t_i)^2} }
\]
where
\[
	\Nweight{i}{t} = {\weight{i}{t} \over \sum_s \weight{i}{s}}~.
\]
\end{itemize}
~\newline
This algorithm is quite different from the Bayes algorithm.
\newline\newline
{\bf Theorem:} For {\em any} sequence the cumulative square loss
of the algorithm is larger by at most ${1\over 2}\ln N$ 
from the loss of the best expert.
\pagebreak

% %%%%%%%%%%%%%%%%%%%%%%%%%%%%%%%%%%%%%%%%%%%%%%%%%%%%%%%%
\begin{center}
\underline{Bayes algorithm is sub-optimal for square loss}
\end{center}

Bayes is optimal only if data is generated by a model
from the class.
\newline
\newline
{\bf Example:}\newline
$N=3$\newline
$\forall t\;\; p_1^t=0,\; p_2^t=1/2,\; p_3^t=1$\newline
Source: biased coin with $p=0.9$.
\newline\newline
For {\em any} prior: Bayes algorithm quickly converges to $p=1/2$.
\newline
\begin{tabbing}
Expected \= total loss of Bayes: \\ \> $ \approx (1/2)^2*T=0.25*T$ \\
Expected total loss of expert 3: \\ \> $(0.9*(1-1)^2+0.1*(1-0)^2)*T=0.1*T$ \\
Expected total loss for Vovk's algorithm: \\ \> $\leq 0.1*T+\ln(3)/2$.
\end{tabbing}
\pagebreak

% %%%%%%%%%%%%%%%%%%%%%%%%%%%%%%%%%%%%%%%%%%%%%%%%%%%%%%%%
\begin{center}
\underline{A prediction algorithm for binary outcome, absolute loss}
\newline [Cesa-Bianchi et. al. 93]
\end{center}
\[ \loss{}{t} =  |x^t - p_i^t| \]

\begin{itemize}
\item
There is no achievable pair $(1,c)$, $c<\infty$.
\item
By selecting $\eta$ as a function of $T$ we can achieve
\[
\TAloss - \BEloss \leq  \sqrt{T \ln (N+1) \over 2} + {\log_2(N+1) \over 2}
\]
\item
{\em Lower bound:} If all predictions and outcome are chosen to be $0$
or $1$ with equal probability  then, for {\em any} algorithm:
\[
\TAloss - \BEloss \geq (1-o(1)) \sqrt{T \ln N \over 2} 
\mbox{ when } T,N \to \infty
\]
\end{itemize}

\pagebreak

% %%%%%%%%%%%%%%%%%%%%%%%%%%%%%%%%%%%%%%%%%%%%%%%%%%%%%%%%
\begin{center}
\underline{A lower bound construction for absolute loss}
\end{center}
\begin{itemize}
\item
{\bf Sequence:} Random, IID, $p={1 \over 2}$.
\item
{\bf $N$ Experts:} Random, IID, $p={1 \over 2}$.
\end{itemize}
~\newline
Expected loss of {\em any} algorithm $= {T \over 2}$.
~\newline
~\newline
Expected loss of {\bf best} expert 
$ = {T \over 2} - (1-o(1)) \sqrt{T \ln N \over 2}$.
when $T, N \to \infty$.
~\newline
\begin{itemize}
\item
$T \to \infty$ : $L'_i = {L_i-T/2 \over \sqrt{T}}$ converges to
normal.
\item
$N \to \infty$ : $\sqrt{2 \ln N}\max(L'_1,\ldots,L'_N)+2 \ln N$
converges to a limit distribution with finite expected value.
\end{itemize}
\pagebreak

% %%%%%%%%%%%%%%%%%%%%%%%%%%%%%%%%%%%%%%%%%%%%%%%%%%%%%%%%
\begin{center}
\underline{Planting the lower bound inside a less trivial case}
\end{center}

\pagebreak
% %%%%%%%%%%%%%%%%%%%%%%%%%%%%%%%%%%%%%%%%%%%%%%%%%%%%%%%%
\begin{center}
\underline{Summary of part I, $N$ experts}
\end{center}
\begin{itemize}
\item
For any bounded loss ``Hedge'' achieves
\[
\TAloss - \BEloss = O(\sqrt{T \ln N})
\]
\item
For any continuous loss Vovk's algorithm achieves the best bounds of the form
\[
\TAloss \leq a \BEloss + c \ln N
\]
\item
For log loss and square loss Vovk achieves
\[
\TAloss - \BEloss = O(\ln N)
\]
\item
For absolute loss, any algorithm satisfies
\[
\TAloss - \BEloss = \Omega(\sqrt{T \ln N})
\]
\end{itemize}

\pagebreak
% %%%%%%%%%%%%%%%%%%%%%%%%%%%%%%%%%%%%%%%%%%%%%%%%%%%%%%%%
\begin{center}
\underline{A continuous class of models}
\end{center}
\begin{itemize}
\item
Each expert corresponds to a biased coin, 
predicts with some fixed $\theta \in [0,1]$.
\item
All values of $\theta$ allowed.
\item
Uncountably infinite set of experts.
\item
Bound of the form 
\[
	\TAloss \leq a \BEloss +c \ln N
\]
is meaningless.
\item
Can we still get a meaningful bound?
\end{itemize}
\pagebreak
% %%%%%%%%%%%%%%%%%%%%%%%%%%%%%%%%%%%%%%%%%%%%%%%%%%%%%%%%
\begin{center}
\underline{Analysis of uncountably infinite experts}
\end{center}
\begin{itemize}
\item 
Replace the initial weight by a density measure $\dweight{\theta}{} =
\dweight{\theta}{1}$, $\int_0^1 \dweight{\theta}{} d\theta=1$.
\item 
{\bf Upper bound} on final total weight holds translates directly:
\[
 \TAloss \leq -c \ln \int_0^1 \dweight{\theta}{} e^{-\eta \TEloss{\theta}^{T+1}} d\theta
\]
\item
We need a new {\bf lower bound} on the final total weight
\item
{\bf Idea:} If $\dweight{\theta}{t}$ is large then $\dweight{p+\epsilon}{t}$
is also large.
\end{itemize}

\withpsfigs{\centerline{\psfig{figure=neighborhood.ps,width=7in}}}

\pagebreak
% %%%%%%%%%%%%%%%%%%%%%%%%%%%%%%%%%%%%%%%%%%%%%%%%%%%%%%%%
\begin{center}
\underline{Rewriting the integral in exponential form}
\end{center}
\begin{itemize}
\item
For log loss and square loss best $\theta$ is empirical distribution
of seq.
\[
	\btheta = {\#\{x^t=1;\;\; 1 \leq t \leq T \} \over T} 
\]
\item
The total loss scales with $T$:
\[
\TEloss{\theta} =
 T \cdot (\btheta \ell(\theta,1) + (1-\btheta)\ell(\theta,0))
 \doteq T \cdot g(\btheta,\theta)
\]
\item
If $a=1$ then $\eta=1/c$ and we can put everything in the exponent:
\begin{eqnarray*}
\TAloss - \BEloss &\leq&
-c \ln \int_0^1 \dweight{\theta}{} e^{-\eta \TEloss{\theta}} d\theta
-c \ln e^{(1/c) \BEloss} \\
&=&
-c \ln \int_0^1 \dweight{\theta}{} e^{-\eta
(\TEloss{\theta}-\BEloss)} d\theta \\
&=&
-c \ln \int_0^1 \dweight{\theta}{} 
e^{-\eta T (g(\btheta,\theta) - g(\btheta,\btheta))} d\theta \\
\end{eqnarray*}
\end{itemize}
\pagebreak
% %%%%%%%%%%%%%%%%%%%%%%%%%%%%%%%%%%%%%%%%%%%%%%%%%%%%%%%%
\begin{center}
\underline{Approximating the area under the peak}
\end{center}
~\newline
Expanding the exponent around $\theta=\btheta$:
\begin{itemize}
\item
First and second term in the expansion of
$g(\btheta,\theta)-g(\btheta,\btheta)$ around $\btheta$ are zero.
\item
Third term gives a quadratic expression in the exponent \newline
$\Rightarrow$ a gaussian.
\end{itemize}
\withpsfigs{\centerline{\psfig{figure=Laplace.ps,width=7in}}}

\begin{eqnarray*}
\lefteqn{\int_0^1  \dweight{\theta}{} 
         e^{-\eta T (g(\btheta,\theta)-g(\btheta,\btheta))} d\theta}\\
\\
&=& \dweight{\btheta}{} \sqrt{-2 \pi c \over 
T \left. {d^2 \over d\theta^2} \right|_{\theta=\btheta} 
(g(\btheta,\theta)-g(\btheta,\btheta))}
+ O(T^{-3/2})
\end{eqnarray*}

\pagebreak
% %%%%%%%%%%%%%%%%%%%%%%%%%%%%%%%%%%%%%%%%%%%%%%%%%%%%%%%%
\begin{center}
\underline{Choosing the optimal prior}
\end{center}
\begin{itemize}
\item
Would like to choose $\dweight{\theta}{}$ to maximize
\[
	\min_{\btheta} \dweight{\btheta}{} \sqrt{-2 \pi c \over 
T \left. {d^2 \over d\theta^2} \right|_{\theta=\btheta} 
(g(\btheta,\theta)-g(\btheta,\btheta))}
\]
\item
Make bound equal for all $\btheta \in [0,1]$ by choosing
\[
\dweight{\btheta}{*} =
{1 \over Z}
\sqrt{\left. {d^2 \over d\theta^2} \right|_{\theta=\btheta} 
(g(\btheta,\theta)-g(\btheta,\btheta)) \over - 2 \pi c}~,
\]
where $Z$ is the normalization factor:
\[
Z =\sqrt{1 \over 2 \pi c}\;\;
\int_0^1 \;\;\sqrt{\left. {d^2 \over d\theta^2} \right|_{\theta=\btheta} 
(g(\btheta,\btheta)-g(\btheta,\theta))} \;\;d\btheta
\]
\item
The bound becomes:
\begin{eqnarray*}
\TAloss - \BEloss &\leq&
-c \ln \int_0^1 \dweight{\theta}{*} 
e^{-\eta T (g(\btheta,\theta) - g(\btheta,\btheta))} d\theta \\ \\
&=&
-c \ln \left( \sqrt{2\pi Z \over T} + O(T^{-3/2}) \right) \\ \\
&=&
{c \over 2} \ln {T \over 2\pi} -{c \over 2} \ln Z + O(1/T)~.
\end{eqnarray*}
\end{itemize}

\pagebreak
% %%%%%%%%%%%%%%%%%%%%%%%%%%%%%%%%%%%%%%%%%%%%%%%%%%%%%%%%
\begin{center}
\underline{Biased coins with log loss}
\end{center}
\begin{itemize}
\item
The exponent in the integral is
\[
g(\btheta,\theta) - g(\btheta,\btheta)
=
\btheta \ln {\btheta \over \theta} + 
(1-\btheta) \ln {1-\btheta \over 1-\theta}
=
D_{KL} (\btheta || \theta)
\]
\item
The second derivative
\[
\left. {d^2 \over d\theta^2} \right|_{\theta=\btheta} 
D_{KL} (\btheta || \theta)
\]
Is called the {\em empirical Fisher information}
\item
The optimal prior:
\[
\dweight{\btheta}{*} ={1 \over Z}
\sqrt{T \left. {d^2 \over d\theta^2} \right|_{\theta=\btheta} 
(g(\btheta,\theta)-g(\btheta,\btheta)) \over - 2 \pi c}~,
\]
is known as {\em Jeffrey's prior} which, for this class, 
is the {\em Dirichlet-$(1/2,1/2)$ prior}.
\end{itemize}

\pagebreak
% %%%%%%%%%%%%%%%%%%%%%%%%%%%%%%%%%%%%%%%%%%%%%%%%%%%%%%%%
\begin{center}
\underline{Bounds for biased coins, log loss}
\end{center}

\begin{itemize}
\item
The bound, for Bayes algorithm using the Dirichlet-$(1/2,1/2)$
distribution,
for $\btheta \in [\epsilon,1-\epsilon]$, is:
\[
\TAloss - \BEloss \leq {1 \over 2} \ln (T+1) 
                     + {1 \over 2} \ln {\pi \over 2}
                     + O(1/T)
\]
\item
Bound asymptotically equal to the min/max bound when $T$ known in advance.
\item
The bound when seq. {\bf generated by a biased coin}, for same
algorithm [Xie and Barron, 95] is:
\[
\max_{\theta}
E(\TAloss - \TEloss{\theta}) 
\leq {1 \over 2} \ln (T) 
   + {1 \over 2} \ln {\pi \over 2e}
   + O(1/T)
\]
\item
The prediction is very simple:
\[
p_t = {\#\{x^t=1;\;\; 1 \leq t \leq T \} + 1/2 \over T+1} 
\]
\end{itemize}

\pagebreak
% %%%%%%%%%%%%%%%%%%%%%%%%%%%%%%%%%%%%%%%%%%%%%%%%%%%%%%%%
\begin{center}
\underline{Biased coins with square loss}
\end{center}

\begin{itemize}
\item
The exponent in the integral is
\[
g(\btheta,\theta) - g(\btheta,\btheta)
=
(\theta-\btheta)^2
\]
\item
The second derivative is a constant:
\[
\left. {d^2 \over d\theta^2} \right|_{\theta=\btheta} 
(\theta-\btheta)^2 = 2
\]
\item
The optimal prior is uniform over $[0,1]$.
\item
The bound for $\btheta \in [\epsilon,1-\epsilon]$, is:
\[
\TAloss - \BEloss \leq {1 \over 4} \ln T 
                     - {1 \over 4} \ln {\pi \over 2}
                     + O(1/T)
\]
\item
{\bf Prediction:} complicated to write, but $O(\log T)$ to compute.
\end{itemize}

\pagebreak
% %%%%%%%%%%%%%%%%%%%%%%%%%%%%%%%%%%%%%%%%%%%%%%%%%%%%%%%%
\begin{center}
\underline{Summary}
\end{center}
~\newline
{\bf $N$ models:} 
\begin{enumerate}
\item
A very simple algorithm to predict almost as well as the best expert.
\item
No probabilistic assumptions about the outcomes or the experts.
\item
Algorithm is identical to Bayes for log-loss.
\item
Algorithm is new and better than Bayes for square loss and absolute
loss.
\end{enumerate}
~\newline
{\bf Biased coins:} 
\begin{enumerate}
\item 
Bayes with Jeffrey's prior is also optimal in the
worst case for log loss.
\item
Uniform prior is best in the worst case for square loss.
\end{enumerate}
\pagebreak

% %%%%%%%%%%%%%%%%%%%%%%%%%%%%%%%%%%%%%%%%%%%%%%%%%%%%%%%%
\begin{center}
\underline{Relation to on-line competitive analysis}
\end{center}

Positive Aspects:
\begin{itemize}
\item
Competitive ratio $\to 1$ as $T \to \infty$.
\item
Good bounds on the competitive {\em difference}.
\end{itemize}

Negative aspects:
\begin{itemize}
\item
Off-line algorithm  restricted to choose one strategy
from a (finite) class.
\item
Loss on each iteration must depend only on action taken on same
iteration.\newline
{\bf Example:} Servicing Page Faults. Loss depends on previous
actions.\newline
Dependence can be weakened by making each 
iteration correspond to a large
number of page faults.
\end{itemize}
\pagebreak

% %%%%%%%%%%%%%%%%%%%%%%%%%%%%%%%%%%%%%%%%%%%%%%%%%%%%%%%%
\begin{center}
\underline{Further work}
\end{center}
Ordered by my familiarity with the subject
\begin{enumerate}
\item
Uncountably infinite classes of experts (parameterized experts).
\item
Observing only the loss of the selected action (multiple arm bandit
problem).
\item
Learning to play repeated matrix games.
\item
Relation to boosting.
\item
Efficient calculation of exponentially many experts.
\item
Relations with universal coding and universal portfolios.
(Gallager, Risannen, Feder, Merhav, Cover).
\item
``Specialists'' - Allowing experts to abstain from predicting.
\item
Allowing the identity of the best expert to change from time to time (non-stationarity).
\item
Competing against the best {\em linear combination} of experts.
\item
Relations with calibration methods (Foster, Vohra).
\end{enumerate}
\pagebreak

% %%%%%%%%%%%%%%%%%%%%%%%%%%%%%%%%%%%%%%%%%%%%%%%%%%%%%%%%
\begin{center}
\underline{More efficient versions}
\end{center}

\begin{itemize}
\item 
Generally, $\Omega(N)$ computation time per iteration.
\item
Some expert classes can be done in $O(\log N)$ time per iteration.
\end{itemize}

\begin{enumerate}
\item
~[Littlestone ??]: ``Winnow'': Expert $=$ a disjunction over $k$ out of
$n$ elements (Time$=O(n)$ instead of $O(n^k)$).
\item
~[Warmuth, Maass ??]: Expert $=$ an indicator function of an
axis-parallel box in $[1\ldots m]^d$.
\item
~[Willems Starkov ??],[Helmbold, Schapire ??]: Expert $=$ a pruning of
a fixed decision tree.
\end{enumerate}
\pagebreak

% %%%%%%%%%%%%%%%%%%%%%%%%%%%%%%%%%%%%%%%%%%%%%%%%%%%%%%%%
\begin{center}
\underline{Restricted feedback}
\end{center}
\begin{itemize}
\item $\loss{i}{t} \in [-1,0]$= loss of action $i$ at time $t$.
\item
Learner chooses action $i_t$.
\item
Learner observes only the loss $\loss{i_t}{t}$.
\item
Exploration vs. exploitation achieved by choosing $i_t$ with
probability
\[
p_i^t = \mu  {\weight{i}{t} \over \sum_{j=1}^K \weight{j}{t}}
        + (1-\mu)
\]
\item
Only weight of selected action is updated
\[ 
\weight{i_t}{t+1} = \weight{i_t}{t} \; 
                    \exp (-\eta \loss{i_t}{t}/p_{i_t}^t)
\]
\item
Upper bound:
\[
E [\TAloss - \BEloss] = O(\sqrt{T K \ln K})
\]
\item
Lower bound: for any algorithm
\[
E [\TAloss - \BEloss] = \Omega(\sqrt{T K})
\]
\end{itemize}
\end{LARGE}
\end{flushleft}

\end{document}



\iffalse
\item
Expand the exponent around $p=\btheta$:
\[
\TEloss{\theta}^{T+1}-\BEloss^{T+1} = 
\TEloss{\btheta}^{T+1}-\BEloss^{T+1}
+ \left. {d \over d\theta} \right|_{\theta=\btheta} \TEloss{\theta}^{T+1}
+ \left. {d^2 \over d\theta^2} \right|_{\theta=\btheta} \TEloss{\theta}^{T+1}
+ o((\theta-\btheta)^2)
\]
\item
\[
\int_0^1 \dweight{\theta}{} e^{-\eta (\TEloss{\theta}^{T+1}-\BEloss^{T+1})} d\theta
= \dweight{\btheta}{} 
\]
\fi

