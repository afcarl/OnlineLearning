% $Header: /data/cvsroot/Courses/OnlineLearning/talks/talk3/talk3.tex,v 1.2 2006/01/18 02:26:18 yfreund Exp $

%\documentclass{beamer}
\documentclass[handout]{beamer}
% This file is a solution template for:

% - Giving a talk on some subject.
% - The talk is between 15min and 45min long.
% - Style is ornate.

% Copyright 2004 by Till Tantau <tantau@users.sourceforge.net>.
%
% In principle, this file can be redistributed and/or modified under
% the terms of the GNU Public License, version 2.
%
% However, this file is supposed to be a template to be modified
% for your own needs. For this reason, if you use this file as a
% template and not specifically distribute it as part of a another
% package/program, I grant the extra permission to freely copy and
% modify this file as you see fit and even to delete this copyright
% notice. 


\mode<presentation>
{
  \usetheme{Montpellier}

  %\setbeamercovered{transparent}
  % or whatever (possibly just delete it)
}

\usepackage{xmpmulti} % package that defines \multiinclude

\usepackage[english]{babel}

\usepackage[latin1]{inputenc}

\usepackage{times}
\usepackage[T1]{fontenc}
% Or whatever. Note that the encoding and the font should match. If T1
% does not look nice, try deleting the line with the fontenc.

\title[log loss] % (optional, use only with long paper titles)
{Lossless compression \\ and \\ cumulative log loss}

\author[Freund] % (optional, use only with lots of authors)
{Yoav Freund}
% - Give the names in the same order as the appear in the paper.
% - Use the \inst{?} command only if the authors have different
%   affiliation.

\institute[Universities of Somewhere and Elsewhere] % (optional, but mostly needed)

\subject{Machine Learning}
% This is only inserted into the PDF information catalog. Can be left
% out. 

% If you have a file called "university-logo-filename.xxx", where xxx
% is a graphic format that can be processed by latex or pdflatex,
% resp., then you can add a logo as follows:

% \pgfdeclareimage[height=0.5cm]{university-logo}{university-logo-filename}
% \logo{\pgfuseimage{university-logo}}



% Delete this, if you do not want the table of contents to pop up at
% the beginning of each subsection:
%% \AtBeginSubsection[]
%% {
%%   \begin{frame}<beamer>
%%     \frametitle{Outline}
%%     \tableofcontents[currentsection,currentsubsection]
%%   \end{frame}
%% }


% If you wish to uncover everything in a step-wise fashion, uncomment
% the following command: 

\beamerdefaultoverlayspecification{<+->}

\newcommand{\newmcommand}[2]{\newcommand{#1}{{\ifmmode {#2}\else\mbox{${#2}$}\fi}}}
\newcommand{\newmcommandi}[2]{\newcommand{#1}[1]{{\ifmmode {#2}\else\mbox{${#2}$}\fi}}}
\newcommand{\newmcommandii}[2]{\newcommand{#1}[2]{{\ifmmode {#2}\else\mbox{${#2}$}\fi}}}
\newcommand{\newmcommandiii}[2]{\newcommand{#1}[3]{{\ifmmode {#2}\else\mbox{${#2}$}\fi}}}

\newcommand{\algfnt}{\bf}

\newmcommand{\ouralg}{{\mbox{\algfnt Hedge}({\eta})}}

\newmcommand{\iter}{T}

\newfont{\cmmib}{cmmib10}
\newcommand{\boldell}{{\mbox{\cmmib \symbol{'140}}}}

\newmcommandi{\costvec}{{\boldell}^{#1}}
\newmcommandii{\cost}{{\ell}^{#1}_{#2}}

\newmcommandi{\rd}{\tilde{#1}}

\newmcommandi{\distvec}{{\bf p}^{#1}}
\newmcommandi{\rddistvec}{\rd{\bf p}^{#1}}
\newmcommandii{\dist}{{p}^{#1}_{#2}}
\newmcommandii{\rddist}{\rd{p}^{#1}_{#2}}

\newmcommandi{\bdistvec}{{\bf q}^{#1}}
\newmcommandii{\bdist}{{q}^{#1}_{#2}}

\newmcommandi{\wtvec}{{\bf w}^{#1}}
\newmcommandi{\rdwtvec}{\rd{\bf w}^{#1}}
\newmcommandii{\wt}{{w}^{#1}_{#2}}
\newmcommandii{\rdwt}{\rd{w}^{#1}_{#2}}

\newcommand{\w}[1]{\makebox[12pt]{{#1}}}
\newcommand{\Rps}{\mbox{\tt R}}
\newcommand{\rPs}{\mbox{\tt P}}
\newcommand{\rpS}{\mbox{\tt S}}
\newcommand{\rpstie}{\w{$\frac{1}{2}$}}
\newcommand{\rpswin}{\w{$0$}}
\newcommand{\rpsloss}{\w{$1$}}

\newmcommand{\decspace}{\Delta}
\newmcommand{\decsym}{\delta}
\newmcommandi{\dec}{\decsym^{#1}}
\newmcommand{\decdistsym}{\cal D}
\newmcommandi{\decdist}{{\decdistsym}^{#1}}

\newmcommand{\simpdistspace}{{\bf \cal S}}
\newmcommand{\domset}{{\rm dom}(\decdistsym)}

\newmcommand{\expdistsym}{{\cal E}}
\newmcommandii{\expdist}{{\expdistsym}^{#1}_{#2}}
\newmcommand{\expdecsym}{{\varepsilon}}
\newmcommandii{\expdec}{\expdecsym^{#1}_{#2}}

\newmcommand{\outspace}{\Omega}
\newmcommand{\outsym}{\omega}
\newmcommandi{\out}{\outsym^{#1}}

%\newmcommandii{\Dkl}{D_{\mbox{kl}}\paren{#1||#2}}
\newmcommandii{\Dkl}{{\rm {KL}}\paren{{#1}\;||\;{#2}}}

\newmcommandi{\sumwts}{\sum_{i=1}^N \wt{#1}{i}}

\newmcommand{\lossalg}{L_A}
\newmcommand{\lossouralg}{{L_{\mbox{\scriptsize\algfnt Hedge}(\eta)}}}
\newmcommand{\lossS}{{L_{\mbox{\scriptsize\algfnt S}}}}
\newmcommandi{\lossi}{L_{#1}}
\newmcommandii{\lossit}{L_{#1}^{#2}}

\newmcommandi{\upbnd}{\tilde{#1}}

\newcommand{\angles}[1]{{\left\langle {#1} \right\rangle}}
\newcommand{\paren}[1]{{\left( {#1} \right)}}
\newcommand{\abs}[1]{{\left| {#1} \right|}}
\newcommand{\ceiling}[1]{{\left\lceil {#1} \right\rceil}}

\newfont{\msym}{msbm10}
\newcommand{\real}{\mbox{\msym R}}

\newmcommand{\updatefcn}{U_\eta}

%% \newtheorem{theorem}{Theorem}	
%% \newtheorem{lemma}[theorem]{Lemma}
%% \newtheorem{corollary}[theorem]{Corollary}
%% \newtheorem{definition}{Definition}

%\newcommand{\proof}{\noindent{\bf Proof:} }
%\newcommand{\example}[1]{{\em Example #1.} }
%\newcommand{\qed}{\rule{0.7em}{0.7em}}

\newcommand{\WeakAlg}{\mbox{\algfnt WeakLearn}}
\newcommand{\Boost}{\mbox{\algfnt AdaBoost}}
\newcommand{\EX}{\mbox{\bf EX}}
\newmcommand{\hf}{h_{{f}}}
\newmcommand{\rdhf}{\rd{h}_{{f}}}
\newmcommand{\hfT}{h^T_{{f}}}
\newmcommand{\ranh}{{b}}

\newmcommand{\conclass}{{\cal C}}

\newmcommand{\badvec}{{\bf b}}
\newmcommandi{\bad}{{b}_{#1}}

%%%%%%%% New commands defined for the game-playing paper

\newmcommand{\hedge}{\algfnt Hedge}
\newmcommand{\play}{\algfnt Play}
\newmcommandi{\Glossvec}{{\bg y}^{#1}}
\newmcommandii{\Gloss}{{y}^{#1}_{#2}}
%\newmcommandi{\action}{{I}_{#1}}
\newmcommandi{\Gdistvec}{{\bf \tilde{p}}^{#1}}
\newmcommandii{\Gdist}{{\teilde{p}}^{#1}_{#2}}

%%%%%%%%%%%%%%%%%%%%%%%%%%%%%%%%%%%%%%%%%%%%%%%%%%%%%
\newmcommand{\Idistvec}{{D}}
\newmcommandi{\Idist}{\Idistvec({#1})}
\newmcommand{\Idistt}{\Idistvec_t}

\newmcommand{\Xdist}{{\cal P}}
\newmcommand{\emp}{\hat{\epsilon}}

\newmcommand{\classpc}{Y}
\newmcommand{\numclass}{k}
\newmcommandii{\prob}{\mbox{\rm Pr}_{#1}\left[{#2}\right]}
\newmcommandii{\exval}{\mbox{\rm E}_{#1}\left[{#2}\right]}

\newmcommand{\lab}{y}
\newmcommand{\ploss}{\mbox{ploss}}
\newmcommandii{\avploss}{\ploss_{#1}({#2})}
\newcommand{\sfrac}[2]{\mbox{$\frac{#1}{#2}$}}

\newcommand{\mboosta}{\mbox{\algfnt AdaBoost.M1}}
\newcommand{\mboostb}{\mbox{\algfnt AdaBoost.M2}}
\newcommand{\mboostr}{\mbox{\algfnt AdaBoost.R}}

\newmcommand{\slos}{\mbox{ploss}}
\newmcommandiii{\sloss}{\slos_{#1}({#2},{#3})}
\newmcommandiii{\avsloss}{\slos_{{#1},{#2}}({#3})}

\newmcommandii{\vwt}{{W}^{#1}_{#2}}

\newcommand{\figline}{\rule{\textwidth}{1pt}}

%\newmcommandi{\1}{{\bf 1}({#1})}
\newmcommandi{\1}{[\![{#1}]\!]}

\newmcommand{\confcn}{\kappa}
\newmcommandi{\erint}{\abs{\int_{y_i}^{h_t(x_i)} {#1} dy}}
%\newmcommandi{\erint}{\int_{\min\{y_i,h_t(x_i)\}}^{\max\{y_i,h_t(x_i)\}}{#1}dy}

\newcommand{\blue}[1]{{\color{blue}{#1}}}
\newcommand{\red}[1]{{\color{red}{#1}}}
\newcommand{\redEq}[1]{{\color{red}{$#1$}}}
\newcommand{\W}{\vec{W}}
\newcommand{\V}{\vec{V}}
\newcommand{\X}{\vec{X}}
\newcommand{\loss}{\vec{\ell}}
\newcommand{\HedgeLoss}{L_{\mbox{\footnotesize Hedge}}}


\begin{document}

%\iffalse %%%%%%%%%%%%%%%%%%%%%%%%%%%%%%%%%%%%%%%%%%%%%%%%%%%%%%%%%%%%%%%%%%

\begin{frame}
  \titlepage
\end{frame}

\begin{frame}
  \frametitle{Outline}
  \tableofcontents[pausesections]
  % You might wish to add the option [pausesections]
\end{frame}

\section{Lossless data compression}

\begin{frame}
\frametitle{The source compression problem}
\begin{itemize}
\item
{\bf Example:}
``There are no people like show people'' \\
\pause 
\R{$\stackrel{\mathrm{encode}}{\rightarrow} x \in \{0,1\}^n$} \\
\pause
\R{$\stackrel{\mathrm{decode}}{\rightarrow}$} ``there are no people like show people''
\item
{\bf Lossless:} Message reconstructed perfectly.
\item
{\bf Goal:} minimize expected length  \R{$E(n)$} of coded message. 
\item
Can we do better than \R{$\lceil \log_2(26) \rceil = 5$} bits per character?
\item
{\bf Basic idea:} Use short codes for common messages.
\item
{\bf Stream compression:} 
\begin{itemize}
\item Message revealed one character at a time.
\item Code generated as message is revealed.
\item Decoded message is constructed gradually.
\end{itemize}
\item Easier than block codes when processing long messages.
\item A natural way for describing a distribution.
\end{itemize}
\end{frame}

\section{The guessing game}

\begin{frame}
\frametitle{The Guessing game}
\begin{itemize}
\item
Message reveraled one character at a time
\item
An algorithm predicts the next character from the revealed part of the message.
\item
If algorithm wrong - ask for next guess.
\item
{\bf Example} 
\begin{columns}
\column[t]{0.1cm}{t\\6}
\column[t]{0.1cm} h \\ 2 
\column[t]{0.1cm}{e\\1}
\column[t]{0.1cm} r \\ 2 
\column[t]{0.1cm} e \\ 1 
\column[t]{0.1cm}   \\ 1 
\column[t]{0.1cm} a \\ 5 
\column[t]{0.1cm} r \\ 2 
\column[t]{0.1cm} e \\ 1 
\column[t]{0.1cm}   \\ 1 
\column[t]{0.1cm} n \\ 4 
\column[t]{0.1cm} o \\ 1 
\column[t]{0.1cm}   \\ 1 
\column[t]{0.1cm} p \\ 5 
\column[t]{0.1cm} e \\ 3 
\end{columns}
\item
Code = sequence of number of mistakes.
\item
To decode use the same prediction algorithm
\end{itemize}

\end{frame}

\section{Arithmetic coding}

\begin{frame}
\frametitle{Arithmetic Coding (background)}
\begin{itemize}
\item Refines the guessing game:
\begin{itemize}
\item In guessing game the predictor chooses \R{order} over alphabet.
\item In arithmetic coding the predictor chooses a \R{Distribution} over alphabet.
\end{itemize}
\item First discovered by Elias (MIT).
\item Invented independently by Rissanen and Pasco in 1976.
\item Widely used in practice.
\end{itemize}
\end{frame}

\begin{frame}
\frametitle{Arithmetic Coding (basic idea)}
\begin{itemize}
\item Easier notation: represent characters by numbers \R{$1 \leq c_t
  \leq | \Sigma |$}. (English: \R{$N \doteq |\Sigma|=26$})
\item message-prefix \R{$c_1,c_2,\ldots,c_{t-1}$} represented by line segment \R{$[l_{t-1},u_{t-1})$}
\item Initial segment \R{$[l_0,u_0) = [0,1)$}
\item After observing \R{$c_1,c_2,\ldots,c_{t-1}$}, predictor outputs \\
\R{$\condp{c_t=1}{c_1,c_2,\ldots,c_{t-1}},\ldots, \condp{c_t=|\Sigma|\;}{c_1,c_2,\ldots,c_{t-1}}$},
\item Distribution is used to partition \R{$[l_{t-1},u_{t-1})$} into \R{$| \Sigma |$} sub-segments.
\item next character \R{$c_t$} determines \R{$[l_t,u_t)$}
\item Code = discriminating binary expansion of a point in \R{$[l_t,u_t)$}.
\end{itemize}
\end{frame}

\begin{frame}
\frametitle{Arithmetic Coding (sequence example)}
\begin{columns}

\begin{column}[T]{4cm}
\begin{itemize}
\item Simplest case.
\item \R{$\Sigma = \{0,1\}$}
\item \R{$\forall t,$ \\ $p(c_t=0)=1/3$ \\ $ p_t(c_t=1)=2/3$}
\item Message = \R{$1111$}
\item Code = \R{$111$}
\item {\bf Technical:} Assume decoder knows that length of message is 4.
\end{itemize}
\end{column}

\begin{column}[T]{5.5cm}
\multiinclude[graphics={width=5cm},format=pdf]{figures/ArithmeticCoding}
\end{column}

\end{columns}
\end{frame}

%\fi %%%%%%%%%%%%%%%%%%%%%%%%%%%%%%%%%%%%%%%%%%%%%%%%%%%%%%%%%%%%%%%%%%%

\section{The performance of arithmetic coding}

\begin{frame}
\frametitle{The code length for arithmetic coding}
\begin{itemize}
\item Given \R{$m$} bits of binary expansion we assume the rest are all zero.
\item Distance between two \R{$m$} bit expansions is \R{$2^{-m}$}
\item If \R{$l_T - u_T \geq 2^{-m}$} then there must be a point \R{$x$} described by \R{$m$} expansion bits such that 
\R{$l_T \leq x < u_T$}
\item Required number of bits is \R{$\lceil -\log_2 (u_T-l_T) \rceil$}.
\item \R{$u_T - l_T = \prod_{t=1}^T \condp{c_t}{c_1,c_2,\ldots,c_{t-1}} \doteq p(c_1,\ldots c_T)$}
\item Number of bits required to code \R{$c_1,c_2,\ldots, c_T$} is
\R{$\lceil - \sum_{t=1}^T \log_2 p_t(c_t) \rceil$}.
\item We call \R{$-\sum_{t=1}^T \log_2 p_t(c_t) = -log_2 p(c_1,\ldots c_T)$} the {\color{blue} Cumulative log loss}
\item Holds for \R{all sequences}.
\end{itemize}
\end{frame}

\section{Source entropy}

\begin{frame}
\frametitle{Expected code length}
\begin{itemize}
\item Fix the messsage length \R{$T$}
\item Suppose the message is \R{generated} at random according to the distribution \R{$p(c_1,\ldots c_T)$}
\item Then the expected code length is 
\R{\begin{eqnarray*}
&& \sum_{c_1,\ldots c_T} p(c_1,\ldots c_T) \lceil -\log_2 p(c_1,\ldots,c_T) \rceil  \\
\pause
&\leq& 1-\sum_{c_1,\ldots c_T} p(c_1,\ldots c_T) \log_2 p(c_1,\ldots,c_T) 
\pause
\doteq 1+H(p_T)
\end{eqnarray*}}
\item \R{$H(p_T)$} is the \R{\bf entropy} of the distribution over
  sequences of length \R{$T$}:
\R{$$
H(p_T) \doteq \sum_{(c_1,\ldots, c_T)} p(c_1,\ldots,c_T) \log
\frac{1}{p(c_1,\l dots,c_T)}
$$}
\item Entropy is the expected value of the cumulative log loss when the messages are generated according to the predictive distribution.
\end{itemize}
\end{frame}

\begin{frame}
\frametitle{Shannon's lower bound}
\begin{itemize}
\item Assume \R{$p_T$} is ``well behaved''. For example, IID.
\item Let \R{$T \to \infty$}
\item \R{$H(p) \doteq \lim_{T \to \infty} \frac{H(p_T)}{T}$} exists and is called the per character entropy of the source \R{$p$}
\item The expected code length for \R{any} coding scheme is at least 
\R{\[
\paren{1-o(1)} H(p_T) = \paren{1-o(1)}\; T\; H(p)
\]}
\item The proof of Shannon's lower bound is not trivial (Can be a
  student lecture).
\end{itemize}
\end{frame}

\section{Other properties of log loss}

\subsection{Unbiased prediction}
\begin{frame}
\frametitle{log loss encourages unbiased prediction}
\begin{itemize}
\item Suppose the source is random and the probability of the next outcome is \R{$\condp{c_t}{c_1,c_2,\ldots,c_{t-1}}$} 
\item Then the prediction that minimizes the log loss is \R{$\condp{c_t}{c_1,c_2,\ldots,c_{t-1}}$}.
\item Note that when minimizing expected number of mistakes, the best prediction in this situation is to put all of the probability on the
most likely outcome.
\item There are other losses with this property, for example, square loss.
\end{itemize}
\end{frame}

\subsection{Other examples for using log loss}

\begin{frame}
\frametitle{Monthly bonuses for a weather forecaster}
\begin{itemize}
\item Before the first of the month assign one dollar to the forecaster's bonus. \R{$b_0=1$}
\item Forecaster assigns probability \R{$p_t$} to rain on day \R{$t$}.
\item If it rains on day \R{$t$} then \R{ $b_t = 2 b_{t-1} p_t$}
\item If it does not rain on day \R{$t$} then \R{ $b_t = 2 b_{t-1} (1-p_t)$}
\item At the end of the month, give forecaster \R{$b_T$}
\item Risk averse strategy: Setting \R{$p_t=1/2$} for all days, guarantees \R{$b_T=1$}
\item High risk prediction: Setting \R{$p_t \in \{0,1\}$} results in Bonus \R{$b_T=2^T$} if always correct, zero otherwise.
\item If forecaster predicts with the true probabilities then 
\R{$$ E(\log b_T) = T - H(p_T) $$} and that is the maximal expected value for \R{$E(\log b_T)$}
\end{itemize}
\end{frame}

\begin{frame}
\frametitle{Horse-race betting}
\begin{itemize}
\item You go to the horse races with one dollar \R{$b_0=1$}
\item \R{$m$} horses compete in each race.
\item Before each race, the odds for each horse are announced:
  \R{$o_t(1),\ldots o_t(m)$} (arbitrary positive numbers)
\item You have to divide {\em all} your money among the different
  horses. \R{$\sum_{j=1}^t \hat{p}_t(j)=1$}
\item The horse \R{$1\leq y_t \leq m$}  is winner of the \R{$t$}th race.
\item After iteration $t$, you have
  \R{$b_t=b_{t-1}\hat{p}_t(y_t)o_t(y_t)$} dollars
\item After $n$ races, you have
  \R{$b_n=\prod_{t=1}^n\hat{p}_t(y_t)o_t(y_t)$} dollars.
\item Taking logs, we get cumulative log loss.
\end{itemize}
\end{frame}

%% \subsection{The yield of investment portfolios}

%% \begin{frame}
%% \frametitle{}
%% \begin{itemize}
%% \item XXX
%% \end{itemize}
%% \end{frame}

\section{universal coding}

\begin{frame}
\frametitle{``Universal'' coding}
\begin{itemize}
\item Suppose there are \R{$N$} alternative predictors / experts.
\item We would like to code almost as well as the best predictor.
\item We would like to make almost as much money as the best expert in hind-site.
\end{itemize}
\end{frame}

\subsection{Two part codes}

\begin{frame}
\frametitle{Two part codes}
\begin{itemize}
\item Send the index of the coding algorithm before the message.
\item Requires \R{$\log_2 N$} additional bits.
\item Requires the encoder to make {\bf two} passes over the data.
\item Is the key idea of {\color{blue}MDL} (Minimal Description Length) modeling.
\begin{itemize}
\item Good prediction model = model that minimizes the total code length
\end{itemize}
\item Often inappropriate because based on {\color{blue} lossless} coding. \\
{\bf Lossy} coding often more appropriate.
\end{itemize}
\end{frame}

\subsection{Combining expert advice for cumulative log loss}

\begin{frame}
\frametitle{Combining predictors adaptively}
\begin{itemize}
\item Treat each of the predictors as an ``expert''.
\item Assign a weight to each expert and reduce it if expert performs poorly.
\item Combine expert predictions according to their weights.
\item Would require only a single pass. Truly online.
\item {\bf Goal:} Total loss of algorithm minus loss of best predictor should be at most \R{$\log_2 N$}
\end{itemize}
\end{frame}

\section{Combining experts in the log loss framework}

\begin{frame}
\frametitle{The log-loss framework}
\begin{itemize}
\item Algorithm \R{$A$} predicts a sequence \R{$c^1,c^2,\ldots, c^T$}
over alphabet \R{$\Sigma = \{1,2,\ldots,k\}$}
\item The prediction for the \R{$c^t$}th is a distribution over \R{$\Sigma$}:\\
\R{$\vp_A^t = \langle p_A^t(1),p_A^t(2),\ldots,p_A^t(k) \rangle$} 
\item When $c^t$ is revealed, the loss we suffer is \R{$-\log p_A^t(c^t)$}
\item The {\color{blue}cumulative log loss}, which we wish to minimize, 
is \R{$\lossalg^T = -\sum_{t=1}^T \log p_A^t(c^t)$}
\item \R{$\lceil \lossalg^T \rceil$} is the code length if \R{$A$} is combined with arithmetic coding.
\end{itemize}
\end{frame}

\begin{frame}
\frametitle{The game}
\begin{itemize}
\item Prediction algorithm \R{$A$} has access to \R{$N$} experts.
\item The following is repeated for \R{$t=1,\ldots,T$}
\begin{itemize}
\item Experts generate predictive distributions: \R{$\vp_1^t,\ldots,\vp_N^t$}
\item Algorithm generates its own prediction \R{$\vp_A^t$}
\item \R{$c^t$} is revealed.
\end{itemize}
\item {\bf Goal:} minimize regret:
\R{\[
-\sum_{t=1}^T \log p_A^t(c^t) + \min_{i=1,\dots,N} \paren{-\sum_{t=1}^T \log p_i^t(c^t)} 
\]}
\end{itemize}
\end{frame}

\section{The online Bayes Algorithm}

\begin{frame}
\frametitle{The online Bayes Algorithm}
\begin{itemize}
\item {\color{blue} Total loss} of expert \R{$i$}
\R{$$L_i^t = - \sum_{s=1}^{t} \log p_i^s(c^s);\;\;\; L_i^0 = 0$$}
\item {\color{blue}Weight} of expert \R{$i$}
\R{$$\wt{t}{i} = \wt{1}{i} e^{-L_i^{t-1}} = \wt{1}{i} \prod_{s=1}^{t-1} p_i^s(c^s)$$}
\item
Freedom to choose initial weights.\\
 \R{$\wt{1}{t} \geq 0$}, \R{$\sum_{i=1}^n \wt{1}{i} = 1$}
\item {\color{blue}Prediction} of algorithm \R{$A$}
\R{\[
\vp_A^t = \frac{\sum_{i=1}^N \wt{t}{i} \vp_i^t}{\sum_{i=1}^N \wt{t}{i}}
\]}
\end{itemize}
\end{frame}

%\subsection{Comparison to \ouralg}

\begin{frame}
\frametitle{The \ouralg Algorithm}
Consider action \R{$i$} at time \R{$t$}
\begin{itemize}
\item Total loss:
\R{$$L_i^t = \sum_{s=1}^{t-1} \ell_i^s$$}
\item Weight:
\R{$$\wt{t}{i} = \wt{1}{i} e^{-\eta L_i^t}$$}
Note freedom to choose initial weight (\R{$\wt{1}{i}$})
\R{$\sum_{i=1}^n \wt{1}{i} = 1$}.
\item
\R{$\eta>0$} is the learning rate parameter. Halving: \R{$\eta \to \infty$}
\item Probability:
\R{$$\dist{t}{i} = \frac{\wt{t}{i}}{\sum_{j=1}^N \wt{t}{i}},\;\;
\pause     \distvec{t} = \frac{\wtvec{t}}{\sum_{j=1}^N \wt{t}{i}}$$}
\end{itemize}
\end{frame}

\section{The performance bound}
%\fi %%%%%%%%%%%%%%%%%%%%%%%%%%%%%%%%%%

\begin{frame}
\frametitle{Cumulative loss vs. Final total weight}

\onslide<1-> Total weight: \R{$W^t \doteq \sum_{i=1}^N \wt{t}{i}$}

\onslide<2-> \R{$$
\frac{W^{t+1}}{W^t}  =  \frac{\sum_{i=1}^N \wt{t}{i} e^{\log p_i^t(c^t)}}{\sum_{i=1}^N \wt{t}{i}} 
\onslide<3->          =   \frac{\sum_{i=1}^N \wt{t}{i} p_i^t(c^t)}{\sum_{i=1}^N \wt{t}{i}} 
\onslide<4->        =  p_A^t(c^t)
$$}
\onslide<5-> \R{$$ -\log \frac{W^{t+1}}{W^t} = -\log p_A^t(c^t) $$}
\R{\[
\onslide<8-> -\log W^{T+1} =
\onslide<6-> -\log \frac{W^{T+1}}{W^1} = -\sum_{t=1}^T \log p_A^t(c^t)
\onslide<7-> = L_A^T
\]}
\onslide<9-> \R{\bf EQUALITY} not bound!
\end{frame}

\begin{frame}
\frametitle{Simple Bound}
\begin{itemize}
\item Use uniform initial weights \R{$\wt{1}{i} = 1/N$}
\item Total Weight is at least the weight of the best expert.
\R{\begin{eqnarray*}
L_A^T & = & -\log W^{T+1} 
\pause = -\log \sum_{i=1}^N \wt{T+1}{i} \\
\pause & = & -\log \sum_{i=1}^N \frac{1}{N} e^{-L_i^T} 
\pause  =  \log N - \log \sum_{i=1}^N e^{-L_i^T}\\
\pause & \leq & \log N - \log \max_i e^{-L_i^T}  
\pause = \log N + \min_i L_i^T
\end{eqnarray*}}
\item Dividing by $T$ we get
\R{$ \frac{L_A^T}{T} = \min_i \frac{L_i^T}{T} + \frac{\log N}{T} $}
\end{itemize}
\end{frame}

%\subsection{Comparison to \ouralg}
\begin{frame}
\frametitle{Upper bound on \R{$\sumwts{\iter+1}$} for \ouralg}
\begin{lemma}[upper bound] 
For any sequence of loss vectors \R{$\costvec{1},\ldots,\costvec{\iter}$}
we have
\R{\[
\ln\paren{\sumwts{\iter+1}} \leq -(1-e^{-\eta}) \lossouralg.
\]}
\end{lemma}
\end{frame}

\begin{frame}
\frametitle{Tuning \R{$\eta$} as a function of \R{$T$}}
\begin{itemize}
\item trivially \R{$\min_i \lossi{i} \leq T$}, yielding
\R{\[
\lossouralg \leq \min_i \lossi{i} + \sqrt{2 T \ln N} + \ln N
\]}
\item per iteration we get:
\R{\[
\frac{\lossouralg}{T} \leq \min_i \frac{\lossi{i}}{T} + \sqrt{\frac{2 \ln N}{T}} + \frac{\ln N}{T}
\]}
\end{itemize}
\end{frame}

%\subsection{Advantage over two part codes}

\begin{frame}
\frametitle{Bound better than for two part codes}
\begin{itemize}
\item
Simple bound as good as bound for two part codes (MDL) 
but enables online compression
\item Suppose we have \R{$K$} copies of each expert.
\item Two part code has to point to one of the $KN$ experts
\R{$ L_A \leq \log NK + \min_i L_i^T = \log NK + \min_i L_i^T$}
\item If we use Bayes predictor + arithmetic coding we get:
\R{$$ L_A = -\log W^{T+1} \leq \log  K \max_i \frac{1}{NK} e^{-L_i^T} 
=\log N + \min_i L_i^T $$}
\item We don't pay a penalty for copies.
\item More generally, the regret is smaller if many of the experts perform well.
\end{itemize}
\end{frame}


\iffalse %%%%%%%%%%%%%%%%%%%%%%%%%%%%%%%%%%%%%%%%%%%%%%%%%%%%%%%%%%%%%%%%%%

\fi %%%%%%%%%%%%%%%%%%%%%%%%%%%%%%%%%%

\end{document}


