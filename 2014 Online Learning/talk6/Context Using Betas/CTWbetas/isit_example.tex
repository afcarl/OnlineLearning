
%% Modified from bare_conf.tex on 06-12-04
%% V1.2
%% 2002/11/18
%% by Michael Shell
%% mshell@ece.gatech.edu
%%
%
\documentclass[10pt,conference]{IEEEtran}
% If the IEEEtran.cls has not been installed into the LaTeX system files,
% manually specify the path to it:
% \documentclass[conference]{../sty/IEEEtran}

\begin{document}

% paper title
\title{Instructions for Preparation and Submission of Papers for the Proceedings of ISIT 2005}


% author names and affiliations
% use a multiple column layout for up to three different
% affiliations
\author{\authorblockN{Thushara Abhayapala}
\authorblockA{Research School of Info. Sci. \& Eng.\\
Australian National University\\
Canberra ACT 0200, Australia \\
Email: thush@syseng.anu.edu.au}
\and
\authorblockN{Leif Hanlen}
\authorblockA{Wireless Signal Processing Program\\
National ICT Australia\\
Canberra ACT 0200, Australia\\
Email: Leif.Hanlen@nicta.com.au} \and
\authorblockN{Alex Grant}
\authorblockA{Inst. for Telecommunications Research\\
University of South Australia\\
Mawson Lakes SA 5095, Australia\\
Email: Alex.Grant@unsa.edu.au}
 }
% avoiding spaces at the end of the author lines is not a problem with
% conference papers because we don't use \thanks or \IEEEmembership
% for over three affiliations, or if they all won't fit within the width
% of the page, use this alternative format:
%
%\author{\authorblockN{Michael Shell\authorrefmark{1},
%Homer Simpson\authorrefmark{2},
%James Kirk\authorrefmark{3},
%Montgomery Scott\authorrefmark{3} and
%Eldon Tyrell\authorrefmark{4}}
%\authorblockA{\authorrefmark{1}School of Electrical and Computer Engineering\\
%Georgia Institute of Technology,
%Atlanta, Georgia 30332--0250\\ Email: mshell@ece.gatech.edu}
%\authorblockA{\authorrefmark{2}Twentieth Century Fox, Springfield, USA\\
%Email: homer@thesimpsons.com}
%\authorblockA{\authorrefmark{3}Starfleet Academy, San Francisco, California 96678-2391\\
%Telephone: (800) 555--1212, Fax: (888) 555--1212}
%\authorblockA{\authorrefmark{4}Tyrell Inc., 123 Replicant Street, Los Angeles, California 90210--4321}}


% make the title area
\maketitle

\begin{abstract}
This paper provides the instructions for the preparation of papers
for submission to ISIT 2005 and relevant style file that produced
this page.
\end{abstract}


\section{Introduction}
The 2005 IEEE International Symposium on Information Theory will be
held at the Adelaide Convention Centre in Adelaide, Australia, from
Sunday, Sept. 4, through Friday, Sept. 9, 2005.



\section{Submission and Review Process}
Papers will be reviewed on the basis of a manuscript ({\bf not
exceeding five pages, two columns, main text font size no smaller than
10 points}) of sufficient detail to permit reasonable evaluation. The
deadline for submission is {\bf January 30, 2005}, with notification
of decisions by {\bf May 15, 2005}. 

The deadline and five page limit will be strictly
enforced. In view of the large number of submissions expected,
multiple submissions by the same author will receive especially
stringent scrutiny. All accepted papers will be allowed twenty
minutes for presentation.

\section{Proceedings}
This year, the ISIT proceedings will be
changing format. Accepted papers will be published in full (up to
five pages in length) on CD-ROM. A hard-copy book of abstracts will
also be distributed at the Symposium to serve as a guide to the
sessions.  

The deadline for the submission of the final camera-ready
paper is {\bf 10 June 2005}. Final manuscript guidelines will be
made available after the notification of decisions.


\section{Preparation of the Paper}
Only electronic submissions in form of a Postscript (PS) or Portable Document
Format (PDF) file will
be accepted. Most authors will prepare their papers with \LaTeX. The
\LaTeX\ style file (\verb#IEEEtran.cls#) and the \LaTeX\ source
(\verb#isit_example.tex#) that produced this page may be downloaded
from the ISIT 2005 web site. Authors using other means to prepare their manuscripts should attempt to duplicate the style of this example as closely as possible.

The {\bf Abstract} section should be no more than $250$ words and
should contain no math notation. If citations are required in the abstract, they should be self-contained, e.g. Shannon, \emph{Bell Syst.\ Tech.\ J.} 1948, rather than [1]. The abstract will be published separately in the hardcopy book of abstracts.

The style of references, e.g.,
\cite{Shannon1948}, equations, figures, tables, etc., should be the
same as for the \emph{IEEE Transactions on Information Theory}. The
affiliation shown for authors should constitute a sufficient mailing
address for persons who wish to write for more details about the
paper.

\section{Electronic Submission}
The paper submission portal will be opened later this year.  Check
\begin{verbatim}
      http://www.isit2005.org/
\end{verbatim}
for detailed instructions. 

% An example of a floating figure using the graphicx package.
% Note that \label must occur AFTER (or within) \caption.
% For figures, \caption should occur after the \includegraphics.
%
%\begin{figure}
%\centering
%\includegraphics[width=2.5in]{myfigure}
% where an .eps filename suffix will be assumed under latex,
% and a .pdf suffix will be assumed for pdflatex
%\caption{Simulation Results}
%\label{fig_sim}
%\end{figure}


% An example of a double column floating figure using two subfigures.
%(The subfigure.sty package must be loaded for this to work.)
% The subfigure \label commands are set within each subfigure command, the
% \label for the overall fgure must come after \caption.
% \hfil must be used as a separator to get equal spacing
%
%\begin{figure*}
%\centerline{\subfigure[Case I]{\includegraphics[width=2.5in]{subfigcase1}
% where an .eps filename suffix will be assumed under latex,
% and a .pdf suffix will be assumed for pdflatex
%\label{fig_first_case}}
%\hfil
%\subfigure[Case II]{\includegraphics[width=2.5in]{subfigcase2}
% where an .eps filename suffix will be assumed under latex,
% and a .pdf suffix will be assumed for pdflatex
%\label{fig_second_case}}}
%\caption{Simulation results}
%\label{fig_sim}
%\end{figure*}


% An example of a floating table. Note that, for IEEE style tables, the
% \caption command should come BEFORE the table. Table text will default to
% \footnotesize as IEEE normally uses this smaller font for tables.
% The \label must come after \caption as always.
%
%\begin{table}
%% increase table row spacing, adjust to taste
%\renewcommand{\arraystretch}{1.3}
%\caption{An Example of a Table}
%\label{table_example}
%\begin{center}
%% Some packages, such as MDW tools, offer better commands for making tables
%% than the plain LaTeX2e tabular which is used here.
%\begin{tabular}{|c||c|}
%\hline
%One & Two\\
%\hline
%Three & Four\\
%\hline
%\end{tabular}
%\end{center}
%\end{table}


\section{Conclusion}
The conclusion goes here.

% conference papers do not normally have an appendix

% use section* for acknowledgement
\section*{Acknowledgment}
% optional entry into table of contents (if used)
%\addcontentsline{toc}{section}{Acknowledgment}
The authors would like to thank various sponsors for supporting their research. 

% trigger a \newpage just before the given reference
% number - used to balance the columns on the last page
% adjust value as needed - may need to be readjusted if
% the document is modified later
%\IEEEtriggeratref{8}
% The "triggered" command can be changed if desired:
%\IEEEtriggercmd{\enlargethispage{-5in}}

% references section
% NOTE: BibTeX documentation can be easily obtained at:
% http://www.ctan.org/tex-archive/biblio/bibtex/contrib/doc/

% can use a bibliography generated by BibTeX as a .bbl file
% standard IEEE bibliography style from:
% http://www.ctan.org/tex-archive/macros/latex/contrib/supported/IEEEtran/bibtex
%\bibliographystyle{IEEEtran.bst}
% argument is your BibTeX string definitions and bibliography database(s)
%\bibliography{IEEEabrv,../bib/paper}
%
% <OR> manually copy in the resultant .bbl file
% set second argument of \begin to the number of references
% (used to reserve space for the reference number labels box)
\begin{thebibliography}{1}


\bibitem{Shannon1948}
C. E. Shannon, ``A mathematical theory of communication,''
\emph{Bell Syst.\ Tech.\ J.}, vol.\ 27, pt.~I, pp.~379--423, 1948;
     pt.~II, pp.~623--656, 1948.


\end{thebibliography}


\end{document}
