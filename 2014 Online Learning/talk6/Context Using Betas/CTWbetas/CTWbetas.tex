%% Modified from bare_conf.tex on 06-12-04
%% V1.2
%% 2002/11/18
%% by Michael Shell
%% mshell@ece.gatech.edu
%%
%
\documentclass[10pt,conference]{IEEEtran}
% If the IEEEtran.cls has not been installed into the LaTeX system files,
% manually specify the path to it:
% \documentclass[conference]{../sty/IEEEtran}

\newtheorem{theorem}{Theorem}
\setcounter{page}{100}


\newcommand{\xpast}{x_{1-D}^{0}}
\newcommand{\xT}{x^{T}_{1}}
\newcommand{\xt}{x^{t}_{1}}
\newcommand{\XT}{X^{T}_{1}}
\newcommand{\Xt}{X^{t}_{1}}
\newcommand{\xtm}{x^{t-1}_{1}}
\newcommand{\xtmW}{x^{t-1}_{t-W}}
\newcommand{\xtWm}{x^{t-W-1}_{1}}
\newcommand{\half}{\frac{1}{2}}
\newcommand{\cS}{{\cal{S}}}
\newcommand{\hcS}{{\hat{\cal{S}}}}
\newcommand{\cTD}{{\cal{T}_{D}}}
\newcommand{\define}{\stackrel{\Delta}{=}}
\newcommand{\alphaexp}{ \frac{ \alpha_{s} }{ 1-\alpha_{s} } }
\newcommand{\GD}{{\Gamma_{D}}}
\newcommand{\apri}{{P^{*}}}

\begin{document}

% paper title
\title{Context-Tree Weighting and Maximizing: \\Processing Betas}

% author names and affiliations
% use a multiple column layout for up to three different
% affiliations

\author{\authorblockN{Frans M.J. Willems}
\authorblockA{Eindhoven University of Technology\\
%Electrical Engineering Department\\
Eindhoven, The Netherlands \\
Email: f.m.j.willems@tue.nl}
\and
\authorblockN{Tjalling J. Tjalkens}
\authorblockA{Eindhoven University of Technology\\
%Electrical Engineering Department\\
Eindhoven, The Netherlands \\
Email: t.j.tjalkens@tue.nl}
\and
\authorblockN{Tanya Ignatenko}
\authorblockA{Eindhoven University of Technology\\
%Electrical Engineering Department\\
Eindhoven, The Netherlands \\
Email: t.ignatenko@tue.nl}
 }
% avoiding spaces at the end of the author lines is not a problem with
% conference papers because we don't use \thanks or \IEEEmembership
% for over three affiliations, or if they all won't fit within the width
% of the page, use this alternative format:
%
%\author{\authorblockN{Michael Shell\authorrefmark{1},
%Homer Simpson\authorrefmark{2},
%James Kirk\authorrefmark{3},
%Montgomery Scott\authorrefmark{3} and
%Eldon Tyrell\authorrefmark{4}}
%\authorblockA{\authorrefmark{1}School of Electrical and Computer Engineering\\
%Georgia Institute of Technology,
%Atlanta, Georgia 30332--0250\\ Email: mshell@ece.gatech.edu}
%\authorblockA{\authorrefmark{2}Twentieth Century Fox, Springfield, USA\\
%Email: homer@thesimpsons.com}
%\authorblockA{\authorrefmark{3}Starfleet Academy, San Francisco, California 96678-2391\\
%Telephone: (800) 555--1212, Fax: (888) 555--1212}
%\authorblockA{\authorrefmark{4}Tyrell Inc., 123 Replicant Street, Los Angeles, California 90210--4321}}


% make the title area
\maketitle

\begin{abstract}
The context-tree weighting method (Willems, Shtarkov, and Tjalkens [1995]) is a sequential universal source coding method that achieves the Rissanen lower bound [1984] for tree sources.
The same authors also proposed context-tree maximizing, a two-pass version of the context-tree weighting method [1993].
Later Willems and Tjalkens [1998] described a method based on ratios (betas) of sequence probabilities that can be used to reduce the storage complexity of the context-tree weighting method.
These betas can be applied to express a posteriori model probabilities in a recursive way (Willems, Nowbahkt-Irani, Volf [2001]).
In the present paper we present new results related to betas.
These results provide a new view on the relation between context-tree weighting and maximizing.
\end{abstract}

%%%%%%%%%%%%%%%%%%%%%%%%%%%%%%%%%%%%%%%%%%%%%%%%%%%%%%%%%%%%%%%%%%%%%%%%
\section{Introduction: Context-tree weighting}
%=======================================================================
\subsection{Arithmetic Coding}
Denote the binary sequence $x_1 x_2 \cdots x_T$ by $\xT$.
Given a coding distribution $P_{c}(\xT)$ over all binary sequences of length $T$, the Elias algorithm (see e.g. Jelinek~\cite{Jel68}) generates codewords that satisfy the prefix condition with lengths
\begin{equation}
 \label{eq:bndLcod}
L(\xT) = \lceil \log_2 \frac{1}{P_{c}(\xT)} \rceil +1 < \log_2 \frac{1}{P_{c}(\xT)}+2.
\end{equation}
Implementations of this method are called arithmetic coding methods (e.g. Rissanen~\cite{Ris76}, Pasco~\cite{Pas76}).
The codeword length that we obtain in this way is at most two binary digits longer than the length that we desire (i.e. the ideal codeword length $-\log_2 P_c(\xT)$).
We say that the individual {\em coding redundancy} is smaller than 2.
Therefore universal source coding is mainly concerned with finding good coding distributions.

%=======================================================================
\subsection{Krichevski-Trofimov estimator}
The actual probability $\Pr\{\Xt=\xt\}$ of a source sequence $\xt$ for $t=1,T$ is denoted as $P_a(\xt)$.
For an independent identically distributed (i.i.d.) binary source with an unknown parameter $\theta = P_a(1)$ we should use
\begin{equation}
 \label{eq:KT}
P_{e}(a,b) = \frac{(a-\half)(a-\frac{3}{2})\cdot \dots \cdot \half (b-\half)(b-\frac{3}{2})\cdot \dots \cdot \half} {(a+b)(a+b-1)\cdot \dots \cdot 1}
\end{equation}
as coding probability for a sequence containing $a$ zeroes and $b$ ones.
This assignment is called the Krichevsky-Trofimov (KT) estimate~\cite{KriTro81}.
Consider a sequence $\xT$ with $a$ zeroes and $b$ ones, then from (\ref{eq:bndLcod}) we may conclude that
\begin{equation}
  \label{eq:bndLonepar}
  L(\xT) < \log_2 \frac{1}{P_{e}(a,b)} + 2.
\end{equation}
Define the individual redundancy for sequence $\xT$ as
\begin{equation}
  \label{eq:defrho}
\rho(\xT) \define L(\xT) - \log_2 \frac{1}{P_a(\xT)},
\end{equation}
then this redundancy for a sequence $\xT$ with $a$ zeroes and $b$ ones satisfies
\begin{eqnarray}
  \label{bndredonepar}
\rho(\xT) &<& \log_2\frac{1}{P_{e}(a,b)} + 2 -
\log_2\frac{1}{(1-\theta)^{a}\theta^{b}} \nonumber \\
&=& \log_2\frac{(1-\theta)^{a}\theta^{b}}{P_{e}(a,b)} + 2 \leq \frac{1}{2}\log_2 T + 3,
\end{eqnarray}
where we used lemma 1 from \cite{WilShtTja95} to upper bound the $\log_2( P_{a}/P_{e} )$-term.
This term, called the {\em parameter redundancy}, is never larger than $\frac{1}{2}\log_2(a+b) + 1$.
Hence the individual redundancy is not larger than $\frac{1}{2}\log_2 T + 3$ for all $\xT$ and all $\theta \in [0,1]$.
Therefore this estimator is asymptotically optimal (see Rissanen~\cite{Ris84}).

%=======================================================================
\subsection{Tree Sources}
\begin{figure}
\centering
\input{modparx.latex}
\caption{Model (suffix set) and parameters.}
\label{fig:tree}
\end{figure}
Consider figure \ref{fig:tree}.
For a {\em tree source} the probability $P_{a}(X_{t}=1|\cdots,x_{t-2},x_{t-1})$ is determined by starting in the root $\lambda$ of the tree and moving along the path $x_{t-1}, x_{t-2}, \cdots$ until a leaf of the tree is reached.
In this leaf $s$ we find the desired probability (parameter) $\theta_{s}$.
The suffix set or tree $\cS,$ containing the paths to all leaves, is called the {\em model} of the source.

For the source in figure \ref{fig:tree} the actual (conditional) probability of the source generating the sequence $01101$ given the past symbols $\cdots 010$ is:
\begin{eqnarray}
P_{a}(01101|\cdots 010)
&=& (1-\theta_{10}) \theta_{00} \theta_{1} (1-\theta_{1}) \theta_{10} \nonumber \\
&=& 0.00945.
\end{eqnarray}

%========================================================================
\subsection{Unknown Parameters, Known Model}\label{subsec:knownmodel}
The source model (tree) $\cS$ partitions the source sequence in i.i.d. sub-sequences, one for each leaf $s\in\cS$.
If the parameters of the source are unknown we can use the KT-estimator for each of these sub-sequences.
E.g. for ${\cal S}= \{ 00, 10,1\}$ we get:
\begin{equation}
P_{e}(\xT|{\cal S}) = P_{e}(a_{00},b_{00}) \cdot P_{e}(a_{10},b_{10})\cdot P_{e}(a_{1},b_{1}),
\end{equation}
where $a_{s}$ is the number of zeroes in the subsequence of $\xT$ corresponding to leaf $s$, and $b_{s}$ the number of ones in this subsequence.
In general we obtain for the estimated probabilities for tree-model ${\cal S}$:
\begin{equation}
P_{e}(\xT|{\cal S}) = \prod_{s\in {\cal S}} P_{e}(a_{s},b_{s}).
\label{eq:codprobknmod}
\end{equation}

If we use this probability estimate as coding probability, we obtain for the (parameter plus coding) redundancy
\begin{eqnarray}
\rho(\xT) & < & \log_{2}\frac{1}{ P_{e}(\xT|{\cal S}) } + 2 - \log_{2}\frac{1}{ P_{a}(\xT) } \nonumber \\
&\leq& \left( \frac{|\cS|}{2} \log_{2}\frac{T}{|\cS|} + |\cS| \right) + 2,
\end{eqnarray}
for $T \geq |\cS|$.
Note that the second inequality follows from the convexity of the $\log_2(\cdot)$.
Moreover $\rho(\xT) < T+2$ for $T<|\cS|$.

%=========================================================================
\subsection{Weighting}
Consider two sources.
For the first source we should use coding distribution $P^{1}_c(\xT)$ to obtain a small redundancy, for the second source we should use distribution $P^{2}_c(\xT)$.
If we need a single code that is good for both sources then
\begin{equation}
P_{w}(\xT) = \frac{P^{1}_c(\xT)+ P^{2}_c(\xT)}{2}
\end{equation}
would be a good coding distribution for this code.
It leads to codeword length
\begin{eqnarray}
L_{w}(\xT)
&<& \log_2 \frac{2}{ P^{1}_c(\xT) + P^{2}_c(\xT) } +2 \nonumber \\
&\leq & \log_2 \frac{1}{ P^{i}_c(\xT) }+ 3 \mbox{, for $i=1,2$},
\end{eqnarray}
and we loose at most one binary digit with this weighting technique!

%========================================================================
\subsection{Unknown Model}
Suppose that the actual source model $\cS$ is unknown, but that its depth is not larger than $D$.
A {\em context} is a string of binary symbols.
Note that to each context $s$, there corresponds a substring of $x_{1}\cdots x_{T}$ of symbols that are produced by the source following this context $s$.
Let $a_{s}$ be the number of zeroes in this subsequence and $b_{s}$ the number of ones.
The structure that contains a node for all contexts $s$ having depth not larger than $D$ is called the {\em context tree} $\cTD$.
A good probability estimate for the subsequence corresponding to a context (node) $s$ at depth $D$ is $P_{w}^{s} = P_{e}(a_{s},b_{s})$.
Now let the depth $d$ of some node $s$ be less than $D$ and assume that we already have good probability estimates for subsequences corresponding to nodes $0s$ and $1s$ at depth $d+1$.
Denote these probability estimates by $P_{w}^{0s}$ and $P^{1s}_{w}$ respectively.
For the subsequence corresponding to $s$ we then have two alternatives.
We can use the KT-estimate $P_{e}(a_{s},b_{s})$ for the entire subsequence corresponding to $s,$ or we can split up this subsequence in two sub-subsequences and use the product $P_{w}^{0s}P_{w}^{1s}$ of the probabilities $P_{w}^{0s}$ and $P_{w}^{1s}$ as estimate.
If we weight these two alternatives we obtain the weighted probability
\begin{equation}\label{def:CTW}
P_{w}^{s} = \left\{\begin{array}{ll}
                \half P_{e}(a_{s},b_{s}) + \half P_{w}^{0s} P_{w}^{1s}
                &\mbox{if depth($s$) $<D$}, \\
               P_{e}(a_{s},b_{s}) &\mbox{otherwise.}
                 \end{array} \right.
\end{equation}
The weighted probability $P_{w}^{\lambda}$ in the root of the context tree can now be used as coding probability for the entire sequence $\xT$.
The method is called the context-tree weighting (CTW) method.
Important is that the weighted probability realized by CTW satisfies
\begin{eqnarray}
P_{w}^{\lambda}
&=& \sum_{\cS} 2^{-\GD(\cS)} \cdot \prod_{s\in\cS} P_{e}(a_s,b_s), \nonumber \\
&\geq& 2^{-\GD(\cS_a)} \cdot \prod_{s\in\cS_a} P_{e}(a_s,b_s), \label{eq:bndcodprobwei}
\end{eqnarray}
where the summation is over all tree models that fit in the context tree $\cTD$, see Lemma 2 in \cite{WilShtTja95}.
The cost of model $\cS$ is defined as
\begin{equation}
\GD(\cS) \define 2|\cS| - 1 - |\{ s\in\cS, \mbox{depth$(s) = D$} \}|, \label{eq:defcost}
\end{equation}
and $\cS_a$ is the actual model.
%=========================================================================
\subsection{Performance}
The individual redundancy $\rho(\xT)$ relative to the actual source for sequence $\xT$ can be upper bounded by
\begin{eqnarray}
\rho(\xT)
&=&  L_w(\xT) - \log_2 \frac{1}{P_{a}(\xT)}  \nonumber \\
&<& \GD(\cS_a) + \frac{|\cS_a|}{2} \log_2 \frac{T}{|\cS_a|} +|\cS_a|+ 2, \label{eq:bndrhowei}
\end{eqnarray}
for $T\geq|\cS_a|$.
Moreover $\rho(\xT) < \GD(\cS_a)+T+2$ for $T<|\cS_a|$.
The three terms in bound (\ref{eq:bndrhowei}) are the cost of specifying the model i.e. $\GD(\cS_a)$, the cost of specifying the parameters which is $\frac{|\cS_a|}{2}\log_2 \frac{T}{|\cS_a|}+|\cS_a|$ and the loss of 2 binary digits due to arithmetic coding.

Observe that bound (\ref{eq:bndrhowei}) also holds for the redundancy relative to {\em any other source tree model $\cS$} with depth $\leq D$.


%%%%%%%%%%%%%%%%%%%%%%%%%%%%%%%%%%%%%%%%%%%%%%%%%%%%%%%%%%%%%%%%%%%%%%%%%%
\section{Ratios of probabilities: betas}
To reduce the complexity we can store in a node instead of the estimated and weighted block probability a {\em probability ratio}.
This idea was proposed in \cite{WilTja98}.
Consider an internal node $s$ in the context tree $\cTD$ and the corresponding {\em conditional} weighted probability $P_{w}^{s}(X_{t}=1|\xtm,\xpast)$.
Assuming that $0s$ (and not $1s$) is a suffix of the context $\xpast,\xtm$ of $x_{t}$, we obtain for this probability that
\begin{eqnarray}
\label{eq:switch}
\lefteqn{ P_{w}^{s}(X_{t}=1|\xtm,\xpast) }  \nonumber \\
&=&\frac{ P_{e}^{s}(\xtm,X_{t}=1) + P_{w}^{0s}(\xtm,X_{t}=1) P_{w}^{1s}(\xtm) }
        { P_{e}^{s}(\xtm)         + P_{w}^{0s}(\xtm        ) P_{w}^{1s}(\xtm) }  \nonumber \\
&=&\frac{ \beta^{s}(\xtm) P_{e}^{s}(X_{t}=1|\xtm) + P_{w}^{0s}(X_{t}=1|\xtm) }
        { \beta^{s}(\xtm) + 1 }
\end{eqnarray}
where
\begin{equation}
\beta^{s}(\xtm)=\frac{ P_{e}^{s}(\xtm) }{ P_{w}^{0s}(\xtm)P_{w}^{1s}(\xtm) }.
\end{equation}
Here we used in the first equality the main CTW-definition (\ref{def:CTW}) and the fact that $P_{w}^{1s}(\xtm,X_{t}=1) = P_{w}^{1s}(\xtm)$ since $1s$ is not a suffix of the context $\xpast,\xtm$.
For simplicity we have omitted $\xpast$ in all conditions.

Assuming that in node $s$ the counts $a_{s}(\xtm)$ and $b_{s}(\xtm)$ are stored, as well as the ratio $\beta^{s}(\xtm)$, leads to the following sequence of operations:
\begin{enumerate}
\item
Node $0s$ delivers the conditional weighted probability $P_{w}^{0s}(X_{t}=1|\xtm)$ to node $s$.
\item
A conditional estimated probability is determined as suggested by Krichevsky and Trofimov \cite{KriTro81}, i.e.:
\begin{equation}
\label{Pefromcnts}
P_{e}^{s}(X_{t}=1|\xtm) = \frac{b_{s}(\xtm)+1/2}{a_{s}(\xtm)+b_{s}(\xtm)+1}.
\end{equation}
The block-version of this KT-estimator appears in (\ref{eq:KT}).
\item
Now the outgoing conditional weighted probability can be computed as described in (\ref{eq:switch}).
\item
The ratio $\beta^{s}(\cdot)$ is then updated with symbol $x_{t}$.
This is done as described below:
\begin{equation}\label{updbeta}
\beta^{s}(\xtm,x_{t}) = \beta^{s}(\xtm) \cdot \frac{ P_{e}^{s} (X_{t}=x_{t}|\xtm) }{ P_{w}^{0s}(X_{t}=x_{t}|\xtm) }.
\end{equation}
\item
Finally, depending on the value $x_t$, either count $a_{s}(\xtm)$ or $b_{s}(\xtm)$ is incremented.
\end{enumerate}

We see that inside the node $s$ there is a {\em switch} that controls the mixture between the incoming conditional weighted probability $P_{w}^{0s}(X_{t}=1|\xtm)$ and the (internal) ratio $P_{e}^{s}(X_{t}=1|\xtm)$.
The mixture is determined by the ratio $\beta^{s}(\xtm)$.
For large $\beta^{s}(\xtm)$ the outgoing conditional probability is roughly $P_{e}^{s}(X_{t}=1|\xtm),$ for small $\beta^{s}(\xtm)$ it is approximately equal to $P_{w}^{0s}(X_{t}=1|\xtm)$.
If $s$ is a leaf of $\cTD$ the outgoing conditional weighted probability is simply $P_{e}^{s}(X_{t}=1|\xtm)$, i.e. the internal one.

A storage complexity reduction is obtained since estimated and weighted block probabilities decrease as the sequence length $T$ increases, while the ratio $\beta^{s}$ corresponds to two different coding alternatives for the subsequence in the node $s$ and is therefore closer to one.
Moreover Eq. (\ref{eq:switch}) shows that in practice the performance does not depend on how large and how small $\beta^{s}$ really can become as long as it is large enough and small enough.
%In \cite{WilTja97} it is shown that good performance can already be obtained for $|\log_2 \beta^s| \leq A$ for $A \approx 20$.

%%%%%%%%%%%%%%%%%%%%%%%%%%%%%%%%%%%%%%%%%%%%%%%%%%%%%%%%%%%%%%%%%%%%%%%%%
\section{On a linear combination}
We can we use equation (\ref{eq:switch}) to express the conditional weighted probability of the root as a linear combination of the estimated probabilities of the nodes along the context path $x_{t-D}x_{t-D+1}\cdots x_{t-1}$.
This results in
\begin{equation}\label{eq:convexcomb}
P_{w}^{\lambda}(\xtm) = \sum_{d=0,D} \mu^{s_d}(\xtm) P_{e}^{s}(X_t=1|\xtm)
\end{equation}
where $s_0\define\lambda$ and $s_d \define x_{t-d} \cdots x_{t-1}$ for $d=1,\cdots,D$, and
\begin{equation}
\mu^{s_{d}}(\xtm) = \frac{\beta^{s_d}(\xtm)}{\beta^{s_d}(\xtm)+1} \prod_{i=0,d-1} \frac{1} {\beta^{s_i}(\xtm)+1},\label{def:mua}
\end{equation}
for $d=0,1,\cdots,D-1$ and
\begin{equation}
\mu^{s_{D}}(\xtm) =  \prod_{i=0,D-1} \frac{1} {\beta^{s_i}(\xtm)+1}.\label{def:mub}
\end{equation}
If we observe that $\mu^{s_d}(\xtm) \geq 0$ for $d=0,1,\cdots,D$ and
\begin{equation}
\sum_{d=0,D} \mu^{s_d}(\xtm) = 1,
\end{equation}
we may conclude that (\ref{eq:convexcomb}) is actually a convex combination.

%%%%%%%%%%%%%%%%%%%%%%%%%%%%%%%%%%%%%%%%%%%%%%%%%%%%%%%%%%%%%%%%%%%%%%%%%
\section{Computing a posteriori model  probabilities}\label{sec:aposteriori}
Consider a sub-tree model $\cS_s$ (a proper and complete set of strings all having a common suffix $s$), rooted in node $s$ of $\cTD$ and fitting in the context tree $\cTD$.
Then we define the "conditional" probability of the sub-tree $\cS_{s}$ given $\xT$ as
\begin{equation}
Q_{w}^{s}(\cS_s) \define \frac{2^{-\GD(\cS_{s})} \prod_{s\in\cS_s} P_e( a_s,b_s) }{ P_{w}^{s} },\label{def:Qw}
\end{equation}
where the cost of sub-model $\cS_{s}$ is defined as
\begin{equation}
\GD(\cS_{s}) \define 2|\cS_{s}| - 1 - |\{ s\in\cS_{s}, \mbox{depth$(s) = D$} \}|. \label{def:subtreecost}
\end{equation}
It makes sense to call this probability a conditional probability since the denominator in (\ref{def:Qw}) can be expressed as
\begin{equation}
P^s_w = \sum_{\cS_{s}} 2^{-\GD(\cS_{s})} \prod_{s\in\cS_s} P_e( a_s,b_s),
\end{equation}
and
\begin{equation}
\sum_{\cS_{s}} 2^{-\GD(\cS_{s})} = 1,
\end{equation}
where the summations are over all sub-models rooted in $s$ having no leaves at depth deeper than $D$.
This is Lemma 2 in \cite{WilShtTja95}.

Now if $|\cS_s|>1$, note that the node $s$ can not be at level $D$ then, we can split up the sub-model $\cS_{s}$ into a sub-model $\cS_{0s}$ and a sub-model $\cS_{1s}$ and we obtain for the conditional probability
\begin{eqnarray}
Q_{w}^{s}(\cS_s) &=&    \frac{ 2^{-\GD(\cS_{0s})}
\prod_{s\in\cS_{0s}}
       P_e(a_s,b_s) } { P_{w}^{0s}} \nonumber \\
&&\cdot\frac{ 2^{-\GD(\cS_{1s})} \prod_{s\in\cS_{1s}}
       P_e(a_s,b_s) } { P_{w}^{1s}} \nonumber \\
&&\cdot\frac{ P_{w}^{0s} P_{w}^{1s} }
       { P_e(a_s,b_s)+P_{w}^{0s}P_{w}^{1s} } \nonumber \\
&=&    Q_{w}^{0s}(\cS_{0s})Q_{w}^{1s}(\cS_{1s})\frac{1}{\beta_{s}+1}.
\end{eqnarray}

When the sub-model $\cS_{s}$ contains only one leaf-node $s$, not at depth $D$, then
\begin{equation}
Q_{w}^{s}(\cS_s)
= \frac{ P_e(a_s,b_s) }{ P_e(a_s,b_s) + P_{w}^{0s}P_{w}^{1s} }
= \frac{\beta_{s}}{\beta_{s}+1}.
\end{equation}

If sub-model $\cS_{s}$ consists only of a single leaf-node $s$ at level $D$ then
\begin{equation}
Q_{w}^{s}(\cS_s) = 1.
\end{equation}

Summarizing the three considered cases we can write
\begin{equation}
\label{eq:defaposteriori}
Q_{w}^{s}(\cS_{s})  =
\left\{\begin{array}{ll}
                Q_{w}^{0s}(\cS_{0s}) Q_{w}^{1s}(\cS_{1s}) \frac{1}{\beta_{s}+1}
                &\mbox{if $|\cS_s|>1$}, \\
                \frac{\beta_{s}}{\beta_{s}+1} \mbox{ if depth$(s)<D$}
                &\mbox{for $|\cS_s|=1$}, \\
                1 \mbox{ if depth$(s)=D$}
                &\mbox{for $|\cS_s|=1$}.
       \end{array} \right.
\end{equation}
If we take
\begin{equation}
P(\cS) \define  2^{-\GD(\cS)}
\end{equation}
as the {\em a priori probability} of model $\cS$, then we can write for the {\em a posteriori probability} of model $\cS$ after having observed $\xT$ that
\begin{equation}
P_{w}(\cS|\xT) = \frac{2^{-\GD(\cS)} \prod_{s\in\cS} P_e(a_s,b_s)
} { P_{w}^{\lambda} } = Q_{w}^{\lambda}(\cS),
\end{equation}
where the last equality follows from (\ref{def:Qw}).
Recursive expression (\ref{eq:defaposteriori}) can now be used to determine the a posteriori probability $Q_{w}^{\lambda}(\cS)$ of a model $\cS$ from the $\beta$'s in the context tree.
We just have to form a product which consists of a factor $1/(\beta_{s'}+1)$ for each internal node $s'$ of the model $\cS$ and a factor $\beta_{s"}/(\beta_{s"}+1)$ for each leaf $s"$ of the model $\cS$ not at level $D$.

%%%%%%%%%%%%%%%%%%%%%%%%%%%%%%%%%%%%%%%%%%%%%%%%%%%%%%%%%%%%%%%%%%%%%%%%%
\section{Introduction: Context-Tree Maximizing}\label{sec:maximizing}
%========================================================================
\subsection{Two-pass methods}
CTW is a {\em one-pass method}.
The source sequence $\xT$ is processed in a sequential way, i.e. the first source symbol $x_1$ is observed, some first code symbols may be produced, the second symbol $x_2$ is observed, more code symbols may be produced, etc. .
In a {\em two-pass system} the entire source sequence $\xT$ is observed first.
Only after that a codeword is constructed.
Consider the following {\em two-pass method}:
\begin{enumerate}
\item
After observing $\xT$ determine the ``best model'' $\hcS$ matching to $\xT$.
\item
Encode this model $\hcS$.
\item
Encode the sequence $\xT$ given this model $\hcS$.
\end{enumerate}
Some questions that arise now are:
(a) What is the best model $\hcS$?
How can it be determined efficiently?
(b) How do we encode the best model $\hcS$ and sequence $\xT$ given model $\hcS$?

%================================================================================
\subsection{The context-tree maximizing algorithm}
When the source produces sequence $\xT$ and model $\cS$ is chosen as the best model, the resulting coding probability\footnote{It satisfies $\sum_{\xT} P^{\lambda}_m(\xT) < 1$ however.} for the two-pass case, is
\begin{equation}
2^{-\GD(\cS)} \cdot \prod_{s\in\cS} P_{e}(a_s,b_s).\label{eq:codprobmax}
\end{equation}
Here the first factor is the number of bits needed to specify the model $\cS$ in a recursive way (i.e. the natural code mentioned in \cite{WilShtTja95}) and the second factor is the coding probability of the sequence $\xT$ given the model $\cS$, see (\ref{eq:codprobknmod}).
The context-tree maximizing (CTM) method, first mentioned in \cite{WilShtTja93} but also proposed by Nohre in his Ph.D. thesis \cite{Noh94}), finds the model maximizing (\ref{eq:codprobmax}) recursively, using a context tree, by taking
\begin{equation}\label{def:CTM}
P_{m}^{s} = \left\{\begin{array}{ll}
                \max [ \half P_{e}(a_{s},b_{s}) , \half P_{m}^{0s} P_{m}^{1s} ]
                &\mbox{if depth($s$) $<D$}, \\
               P_{e}(a_{s},b_{s}) &\mbox{otherwise.}
                 \end{array} \right.
\end{equation}
We assumed that the entire sequence $\xT$ was processed into the context tree.
We finally will find the best model ${\cal S}$ by tracking the maximization procedure, starting in the {\em root} $\lambda$ of the context tree.
If in a node $s$ in the context tree $P_{e}(a_{s},b_{s}) \geq P_{m}^{0s} P_{m}^{1s}$ then $s$ is a leaf of the best tree ${\hcS}$ and we do not have to investigate the sub-tree rooted in $s$ any further.
Otherwise $s$ is an internal node of the best model and we have to check the nodes $0s$ and $1s$.
Note that
\begin{equation}
P_{m}^{\lambda} = \max_{\cS} 2^{-\GD(\cS)} \prod_{s\in\cS} P_{e}(a_s,b_s),
\end{equation}
and denote that model maximizing this expression by $\hcS$.


%=========================================================================
\subsection{Performance}
The coding probability for context-tree maximizing satisfies
\begin{equation}
P_{m}^{\lambda} =
2^{-\GD(\hcS)} \prod_{s\in\hcS} P_{e}(a_s,b_s) \geq 2^{-\GD(\cS_a)}  \prod_{s\in\cS_a} P_{e}(a_s,b_s),
\end{equation}
just like $P_w^{\lambda}$, see (\ref{eq:bndcodprobwei}), and therefore maximizing, just like weighting, leads to the redundancy bound (\ref{eq:bndrhowei}).
Observe that this bound holds for any model, not only for $\cS_a$.

%%%%%%%%%%%%%%%%%%%%%%%%%%%%%%%%%%%%%%%%%%%%%%%%%%%%%%%%%%%%%%%%%%%%%%%%
\section{Finding the maximum a posteriori model}

Observe first that the context-tree maximizing method yields the maximum a posteriori (MAP) tree model given the observed sequence $\xT$.
This observation can be found in \cite{WilNowVol02}.
In \cite{WilNowVol02} also the method for computing a posteriori model probabilities, that was described in section \ref{sec:aposteriori}, was proposed.
It is a bit strange that on one hand a posteriori model probabilities can be computed from the $\beta$'s in a context tree while on the other hand to determine the MAP tree-model we need the context-tree maximizing method.
Therefore we want to find a method that determines the MAP-model based on the $\beta$'s in the weighted context-tree.
First we determine the probability of the MAP sub-model corresponding to a node $s$ at depth $<D$.
For such a node we can write
\begin{eqnarray}
\lefteqn{\max_{\cS_s} Q_{w}^{s}(\cS_{s})} \nonumber \\
&=& \max [ \frac{1}{\beta_{s}+1} \max_{\cS_{0s}} Q_{w}^{0s}(\cS_{0s}) \max_{\cS_{1s}} Q_{w}^{1s}(\cS_{1s}) , \frac{\beta_s}{\beta_{s}+1} ].  \nonumber \\
&&
\end{eqnarray}
Here the last term corresponds to the sub-model which has only a single leaf-node at $s$.
The first term correspond to all larger sub-models.

For a node at depth $D$ only the one-leaf sub-model plays a role and
\begin{equation}
\max_{\cS_s} Q_{w}^{s}(\cS_{s}) = 1.
\end{equation}

If we now define for all nodes $s\in\cTD$ the MAP sub-model probability
\begin{equation}
Q_{mw}^{s} \define  \max_{\cS_s} Q_{w}^{s}(\cS_{s}),
\end{equation}
then the following recursive equation holds:
\begin{equation}
\label{def:MAP}
Q_{mw}^{s} =
\left\{\begin{array}{ll}
                \max [ Q_{mw}^{0s} Q_{mw}^{1s} \frac{1}{\beta_{s}+1} , \frac{\beta_{s}}{\beta_{s}+1} ] &\mbox{if depth$(s)<D,$} \\
                1 &\mbox{if depth$(s)=D.$}
                       \end{array} \right.
\end{equation}
Now in the root $\lambda$ of the context tree we find the maximum a posteriori model probability $Q_{mw}^{\lambda}$.
Tracking the procedure starting in the root of the context tree yields the MAP-model.


%%%%%%%%%%%%%%%%%%%%%%%%%%%%%%%%%%%%%%%%%%%%%%%%%%%%%%%%%%%%%%%%%%
\section{Difference between CTW and CTM}
Let $\hcS$ be the MAP model, then
\begin{equation}
1 \geq \frac{2^{-\GD(\hcS)} \prod_{s\in\hcS} P_e( a_s,b_s) }{ P_{w}^{\lambda} }
= Q_{w}^{\lambda}(\hcS) = Q^{\lambda}_{mw}.
\end{equation}
For the difference in codeword lengths for CTW and CTM we first can write
\begin{eqnarray}
\lefteqn{L_w(\xT)-L_m(\xT)} \nonumber \\
&= & \lceil \log_2\frac{1}{P^{\lambda}_w(\xT)}\rceil + 1 - \lceil\log_2\frac{1}{P_m^{\lambda}(\xT)}\rceil -1 \nonumber \\
&\leq& 0.
\end{eqnarray}
However we can also show that
\begin{eqnarray}
\lefteqn{L_w(\xT)-L_m(\xT)} \nonumber \\
&< & \log_2\frac{1}{P^{\lambda}_w(\xT)} + 2 - \log_2\frac{1}{P^{\lambda}_m(\xT)} - 1 \nonumber \\
&=& \log_2\frac{2^{-\GD(\hcS)} \prod_{s\in\hcS} P_e( a_s,b_s) }{ P_{w}^{\lambda} } +1 \nonumber \\
&=& \log_2 Q_{mw}^{\lambda} + 1,
\end{eqnarray}
and similarly
\begin{equation}
L_w(\xT)-L_m(\xT) > \log_2 Q_{mw}^{\lambda} - 1.
\end{equation}

Note that only for $Q_{mw}^{\lambda} \approx 1$ we obtain that $L_m(\xT) \leq L_w(\xT) + 1$ and CTM yields roughly the same performance as CTW.
In \cite{WilShtTja00} it was shown that for tree sources that fit in the context tree $\cTD$ MAP proability $Q_{mw}^{\lambda} \rightarrow 1$ for asymptotically large sequence length $T$.
When $Q_{mw}^{\lambda}$ is not close to one (no convergence) $L_m(\xT)$ is significantly larger than $L_w(\xT)$.

%%%%%%%%%%%%%%%%%%%%%%%%%%%%%%%%%%%%%%%%%%%%%%%%%%%%%%%%%%%%%%%%%%
\section{A posteriori node probabilities}
We can define the a posteriori {\em node} probability of a node $s \in \cTD$ as
\begin{equation}
Q_w(s) \define \sum_{\cS: s\in\cS} Q_{w}^{\lambda}(\cS),
\end{equation}
where the summation is over all models $\cS$ that contain leaf $s$.

It now can be shown that for all $s\in\cTD$
\begin{equation}
\mu^{s} = Q_{w}(s),
\end{equation}
where $\mu_s$ is as in (\ref{def:mua}) and (\ref{def:mub}).

For a proof of this statement we refer to figure \ref{fig:pathtoleaf}.
Let node $s$ have depth $d$ and assume that $s=u_{d},u_{d-1},\cdots,u_{1},$ a binary string.
Then define $s(i) \define u_{i},u_{i-1},\cdots,u_{1}$ and $s^{c}(i) \define 1-u_{i},u_{i-1},\cdots,u_{1}$ for $i=1,\cdots,d$ and $s(0)\define\lambda$.
The crucial part of the proof is (\ref{eq:defaposteriori}).
Note that the nodes $s(i), i=0,1,\cdots,d-1$ are inner nodes of all models that have a leaf $s$.
These nodes give rise to the factors in (\ref{def:mua}) and (\ref{def:mub}).
Moreover we use that
\begin{equation}
\sum_{\cS_{s^{c}(i)}} Q_{w}^{s^{c}(i)}(\cS_{s^{c}(i)}) =1.
\end{equation}
for all $i=1,2,\cdots,d$.
\begin{figure}
\centering
\input{pathtoleaf.latex}
\caption{A leaf at depth $d$ and the path leading to this leaf.}
\label{fig:pathtoleaf}
\end{figure}

%%%%%%%%%%%%%%%%%%%%%%%%%%%%%%%%%%%%%%%%%%%%%%%%%%%%%%%%%%%%%%%%%%%%%%%%%
\section{Concluding remarks}
We have presented here several new results related to $\beta$'s.
The results are all based on uniform weighting, i.e. using coefficients $\half,\half$ if we weight two alternatives.
Our results carry over to the case where arbitrary coefficients are used.
Note that a uniform distribution over all tree-models can be achieved by a special kind of non-uniform weighting (see \cite{WilShtTja95}).

%%%%%%%%%%%%%%%%%%%%%%%%%%%%%%%%%%%%%%%%%%%%%%%%%%%%%%%%%%%%%%%%%%%%%%%%%%
\begin{thebibliography}{1}
\bibitem{Jel68} F. Jelinek, {\em Probabilistic Information Theory,} New York: McGraw-Hill, 1968, pp. 476 - 489.

\bibitem{KriTro81} R.E. Krichevsky and V.K. Trofimov, ``The Performance of Universal Encoding,'' {\em IEEE Trans. Inform. Theory,} vol.~IT-27, pp. 199 - 207, March 1981.

\bibitem{Noh94} R. Nohre, {\em Some Topics in Descriptive Complexity,} Ph.D. thesis, Linkoping University, Sweden, 1994.

\bibitem{Pas76} R. Pasco, {\em Source Coding Algorithms for Fast Data Compression,} Ph.D. thesis, Stanford University, 1976.

\bibitem{Ris76} J. Rissanen, ``Generalized Kraft Inequality and Arithmetic Coding,'' {\em IBM J. Res. Devel.}, vol. 20, p. 198, 1976.

\bibitem{Ris84} J. Rissanen, ``Universal Coding, Information, Prediction, and Estimation,'' {\em IEEE Inform. Theory,} vol.~IT-30, pp. 629 - 636, July 1984.

\bibitem{WilNowVol02} F.M.J. Willems, A. Nowbahkt-Irani, and P.A.J. Volf, "Maximum A Posteriori Tree Models," {\em Proc. 4th Int. ITG Conf. Source and Channel Coding,} Berlin, Germany, pp. 335 - 340, February 28-30, 2002.

\bibitem{WilShtTja93} F.M.J. Willems, Y.M. Shtarkov and Tj.J. Tjalkens, ``Context Weighting: General Finite Context Sources,'' {\em Proc. I14th Symp. on Inform. Theory in the Benelux,} pp. 120 - 127, Veldhoven, May 17 \& 18, 1993.

\bibitem{WilShtTja95} F.M.J. Willems, Y.M. Shtarkov and Tj.J. Tjalkens, ``The Context-Tree Weighting Method: Basic Properties,'' {\em IEEE Trans. Inform. Theory,} vol. IT-41, pp. 653 - 664, May 1995.

\bibitem{WilShtTja00} F.M.J. Willems, Y.M. Shtarkov and Tj.J. Tjalkens, ``Context-Tree Maximizing,'' {\em 2000 Conf. on Inform. Sciences ans Syst.,} Princeton Univ., Princeton, NJ, TP6.7 - TP6.12, March 15 - 17, 2000.

\bibitem{WilTja98} F.M.J. Willems and Tj.J. Tjalkens, ``Reducing complexity of the context-tree weighting method,'' {\em Proc. IEEE International Symposium on Information Theory,} p. 347, Cambridge, Mass., August 16 - 21, 1998.
%%%%%%%%%%%%%%%%%%%%%%%%%%%%%%%%%%%%%%%%%%%%%%%%%%%%%%%%%%%%%%%%%%%%%%%%%
\end{thebibliography}
\end{document}


