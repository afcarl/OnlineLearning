\documentclass[12pt]{article}

\usepackage{amsfonts}
\usepackage{amssymb,amsmath}
\usepackage{fullpage}
\usepackage{epsfig}
\usepackage{color}
\setlength{\oddsidemargin}{0pt}
\setlength{\evensidemargin}{0pt}
\setlength{\textwidth}{6.0in}
\setlength{\topmargin}{0.7in}
\setlength{\textheight}{8.5in}

\title{Homework 4}

\begin{document}
\maketitle
\begin{enumerate}
\item {\bf Nash Equilibria} \\

We are looking for the Nash equilibrium of the reward matrix
(interpreting this matrix as a loss matrix yields similar results)
\begin{center} 
\begin{tabular}{ |c|c|c|} 
\hline 0,1 & 2,0 & 0,3 \\ 
\hline 3,0 & 0,2 & 1,0 \\
\hline
\end{tabular} 
\end{center}

The first observation is that the leftmost column can be ignored
because from the column player's point of view it is always dominated
by the rightmost column. Thus the column player will assign the left
column probability zero, and this column can be ignored.
We are therefor left with the 2x2 matrix:
\begin{center} 
\begin{tabular}{ |c|c|c|} 
\hline 2,0 & 0,3 \\ 
\hline 0,2 & 1,0 \\
\hline
\end{tabular} 
\end{center}
By considering all 4 combinations of pure strategies one can verify
that none of them is a Nash equilibrium.

In order to consider mixed strategies we assign probabilities $p$ and
$1-p$ to the two rows and probabilities $q$ and $1-q$ to columns 2 and
3.

In order for the row player to be at equilibrium with a mixed strategy
the expected gain of the row player from both rows should be
equal. This implies that $2q=1-q$, i.e. $q=1/3$. Similarly, to make
the two columns have equal expected reward implies that $3p=2(1-p)$,
i.e. $p=2/5$. Therefor the only nash equilibrium assigns to the rows
probabilities $(2/5,3/5)$ and to the columns probabilities
$(0,1/3,2/3)$

\item {\bf Question 7.4: Nash equilibria vs. Correlated equilibria}
Using an arguments similar to the ones used above we find that the two
pure nash equilibria are row 1, column 2 and column 2, row 1. The
mixed-strategy Nash equilibrium is Row:$(2/3,1/3)$, and Col:$(2/3,1/3)$.

Note that the expected loss vectors associated with these three
equilibria are $(0,5)$, $(5,0)$, $(4,4)$. (the first component in each
vector is the expected loss of the row player, the second is the
expected loss of the column player).

The proposed joint distribution has an expected loss vector of
$(2,2)$. It is a correlated equilibrium because both the row player
and the column player would increase their expected loss to $7/3$ if
they play only one of their two actions.

This correlated equilibrium is not in the convex hull of the Nash
equilibria as evident from the fact that the loss vectors are linear
functions of the strategies but the loss vector $(2,2)$ is
not in the convex hull of the vectors $(0,5)$, $(5,0)$, $(4,4)$

\item {Question 7.8}
The basic bound we have on the expected loss of the column player 
for the exponential weights algorithm with learning rate $\eta$, is
\[
\sum_{t=1}^n \ell(p_{t},J_t) \leq 
\min_{i=1,\ldots,N} \sum_{t=1}^n \ell(i,J_t) + {\ln N \over \eta} + {n \eta \over 8} 
\]
holds for every game sequence. Dividing both sides by $n$ we get
\begin{eqnarray} 
{1 \over n} \sum_{t=1}^n \ell(p_{t},J_t) &\leq& 
\min_{i=1,\ldots,N} \frac{1}{n} \sum_{t=1}^n \ell(i,J_t) + {\ln N
  \over n \eta} + {\eta \over 8} \nonumber \\
&\leq& \frac{1}{n} \sum_{t=1}^n v_i \ell(i,J_t)
\leq V + {\ln N \over n \eta} + {\eta \over 8} \label{eqn:hedge}
\end{eqnarray}
where $<v_1,\ldots,v_N>$ is a mixed strategy for the row player which
achieves the game value $v$.

We now bound the difference between the expected loss and the
actual loss when an action $I_t$ is picked according to the
distribution $p_t$. 

The sequence of sums 
\[
S_n \doteq \sum_{i=1}^n (\ell(p_t,J_t) - \ell(I_t,J_t))
\]
Forms a bounded step martingale, We can therefore use Hoeffding /
Azuma to bound this sum. 
\begin{equation} \label{eqn:Hoeffding}
P\left( \sum_{i=1}^n \ell(I_t,J_t) \geq \sum_{i=1}^n \ell(p_t,J_t)
  +\epsilon \right)
\leq e^{-2n\epsilon^2}
\end{equation}

Combining Equations~(\ref{eqn:hedge}) and~(\ref{eqn:Hoeffding}) we get
the desired result.

\end{enumerate}
\end{document}


