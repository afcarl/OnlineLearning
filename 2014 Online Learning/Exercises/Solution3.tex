\documentclass[12pt]{article}

\usepackage{amsfonts}
\usepackage{amssymb,amsmath}
\usepackage{fullpage}
\usepackage{epsfig}
\usepackage{color}
\setlength{\oddsidemargin}{0pt}
\setlength{\evensidemargin}{0pt}
\setlength{\textwidth}{6.0in}
\setlength{\topmargin}{0.7in}
\setlength{\textheight}{8.5in}

\title{Solution Of Homework 3}

\begin{document}
\maketitle
\begin{enumerate}
\item {\bf Exercise 4.8} \\

  Denote the actions by $\{1,\ldots,N\}$. Internal regret takes the
  maximum over all pairs of actions. Swap regret takes the maximum
  over all functions from the set of actions into itself. To prove the
  desired bound we partition the function $\sigma$ which achieves the highest
  smap regret  into $N$ action pairs.

Formally, let $R_{(i,j),n}$ be the regret associated with (not) taking
  action $j$ each time we took action $i$, and $R_{\sigma,n}$ be the
  regret associate with the mapping $\sigma$.

As the sequences of times at which actions $i \neq j$ have been taken
are disjoint, we get that As the times at which different actions were
taken we get that, for any mapping $\sigma$
\[
R_{\sigma,n} = \sum_{i=1}^N R_{(i,\sigma(i)),n} \leq N \max_{i,j} R_{(i,j),n}
\]

From which it follows that 
\[
\max_\sigma R_{\sigma,n} \leq N \max_{i,j} R_{(i,j),n}
\]

\item {\bf Exercise 4.9} \\

Intuitively, the reason that a deterministic predictor cannot be well
calibrated for any sequence is that an adversary can construct the
next bit in the sequence $y_t$ as a function of the deterministically
predetermined prediction $q_t$.

We want to show non $\epsilon$-calibration for $\epsilon<1/3$.

We consider the following strategy for the adversary:
\begin{itemize}
\item If $q_t<2/3$ then $y_t=1$
\item If $q_t\geq 2/3$ then $y_t=0$
\end{itemize}

To prove that the resulting sequence will not be
$\epsilon$-calibrated we need to show the existance of $x$ such that 
$q_t \in [x-\epsilon,x+\epsilon]$ occurs infinitely often and
$| \rho_n^{\epsilon}(x)-x | > \epsilon$

Let $m$ be the number of values of $t$ such that $q_t<2/3$ and
therefor, by construction $y_t=1$. We consider two cases, conditioned
on whether or not $m$ is finite.

If $m$ is infinite, then there must be an $x \leq 2/3-\epsilon$ such
that $q_t \in [x-\epsilon,x+\epsilon]$ infinitely often. For this $x$,
$\rho_n^{\epsilon}(x)=1$ and thus $\limsup_{n \to
  \infty}|x-\rho_n^{\epsilon}(x)| \geq 1/3 + \epsilon > \epsilon$.

If $m$ is finite, the number of times that $y_t=1$ is finite. Therefor
$\lim_{n \to \infty} \rho_n^{\epsilon}(x)=0$. On the other hand, there
must be some $x \geq 2/3$ for which $q_t \in (x-\epsilon,x+\epsilon)$
infinitely often, as $\lim_{n \to \infty} \rho_n^{\epsilon}(x)=0$ for
all $x$, we get that in this case there is also no $\epsilon$-calibration.

\item {\bf Exercise 4.10}\\

We are given that $\lim_{n \to \infty} {1 \over n} \sum_{t=1}^n y_t =a$.

We are using the estimator $q_t = {1 \over t-1} \sum_{s=1}^{t-1} y_t$

We wish to show that the estimator $q_t$ is well calibrated.

Fix $\epsilon$. We consider three cases, $|x-a| > \epsilon$, $|x-a| < \epsilon$,
$|x-a| = \epsilon$.

If $|x-a| > \epsilon$ then there comes a point in the sequence after which
$|q_t-a| < |x-a| - \epsilon$ and so $q_t \notin
(x-\epsilon,x+\epsilon)$. In other words, there are only a finite
number of elements assigned to that $x$ and we don't care whether the
estimator is claibrated for that value of $x$

If $|x-a| < \epsilon$ then there comes a point in the sequence after
which $|q_t - a| < \epsilon - |x-a|$. From that it follows
that $q_t \in (x-a,x+a)$ for all $t>t_x$. Therefor
$\rho_n^{\epsilon}(x) \to a$ and $\limsup | \rho_n^\epsilon (x)-x |
\leq \epsilon$.

However, there seems to be a problem with the proof when $|x-a|
=\epsilon$. In this case it is possible that $q_t \in (x-a,x+a)$ an
infinite number of times and at the same time $q_t \notin (x-a,x+a)$
an intinite number of times.

In fact, it is not hard to construct a sequence for which the claim fails:

Consider the alternating sequence 
\[
0,1,0,1,0,1,\ldots
\]
This is a ligitimate sequence as the average of the prefixes converges
to $1/2$. Let $x=0.49$, $\epsilon=0.01$. In this case $q_t=1/2$ for
odd values of $t$ and $q_t$ for even values of $t$. Thus the
subsequence that corresponds to $q_t \in (0.48,0.5)$ correspings to
$y_t=1$ and the estimate $\rho_t^{\epsilon}(0.49)=1$, i.e. the
estimator is not claibrated!

\end{enumerate}
\end{document}


