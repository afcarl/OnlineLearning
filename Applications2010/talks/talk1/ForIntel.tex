% $Header: /data/cvsroot/Courses/OnlineLearning/talks/talk1/talk1.tex,v 1.5 2006/01/11 03:23:33 yfreund Exp $

\documentclass{beamer}
%\documentclass[handout]{beamer}
\mode<presentation>
{
  \usetheme{Montpellier}

  %\setbeamercovered{transparent}
  % or whatever (possibly just delete it)
}

\usepackage{xmpmulti} % package that defines \multiinclude
\usepackage[english]{babel}
\usepackage[latin1]{inputenc}
\usepackage[T1]{fontenc}
\usepackage{textcomp}

\newcommand{\blue}[1]{{\color{blue} #1}}
\newcommand{\red}[1]{{\color{red} #1}}

\usepackage{times}
\usepackage[T1]{fontenc}
% Or whatever. Note that the encoding and the font should match. If T1
% does not look nice, try deleting the line with the fontenc.

\title%[short title] (optional, use only with long paper titles)
{Adaptive Resource Allocation\\ Intel visit, May 16 2011}

\author[Freund] % (optional, use only with lots of authors)
{Yoav Freund}
% - Give the names in the same order as the appear in the paper.
% - Use the \inst{?} command only if the authors have different
%   affiliation.

%\institute[Universities of Somewhere and Elsewhere] % (optional, but mostly needed)

%\subject
% This is only inserted into the PDF information catalog. Can be left
% out. 

% If you have a file called "university-logo-filename.xxx", where xxx
% is a graphic format that can be processed by latex or pdflatex,
% resp., then you can add a logo as follows:

% \pgfdeclareimage[height=0.5cm]{university-logo}{university-logo-filename}
% \logo{\pgfuseimage{university-logo}}



% Delete this, if you do not want the table of contents to pop up at
% the beginning of each subsection:
%% \AtBeginSubsection[]
%% {
%%   \begin{frame}<beamer>
%%     \frametitle{Outline}
%%     \tableofcontents[currentsection,currentsubsection]
%%   \end{frame}
%% }


% If you wish to uncover everything in a step-wise fashion, uncomment
% the following command: 

\beamerdefaultoverlayspecification{<+->}

\newcommand{\W}{\vec{W}}
\newcommand{\V}{\vec{V}}
\newcommand{\X}{\vec{X}}
\newcommand{\loss}{\vec{\ell}}
\newcommand{\HedgeLoss}{L_{\mbox{\footnotesize Hedge}}}

\begin{document}

\begin{frame}
  \titlepage
\end{frame}

\begin{frame}
  \frametitle{Outline}
  \tableofcontents
  % You might wish to add the option [pausesections]
\end{frame}

%\iffalse %%%%%%%%%%%%%%%%%%%%%%%%%%%%%%%%%%%%%%%%%%%%%%%%%%%%%%%%%%%%%%%%%%
\section{The Problem}
\begin{frame}
\frametitle{Data-center resource allocation}
\begin{itemize}
\item 10,000 or more computers.
\item Massive consumption of power and cooling.
\item Massive communication load.
\item Diverse work loads and Service Level Agreements, Low margins.
\item Diverse hardware.
\item Manual optimization does not scale.
\end{itemize}
\end{frame}


\end{document}


% \iffalse
% \begin{frame}

% \frametitle{The Perceptron Problem}
% \begin{columns}
% \begin{column}[T]{5.2cm}
% \multiinclude[graphics={width=7cm},format=pdf]{perceptronAnim/Perceptron}
% \end{column}

% \begin{column}[T]{4cm}
% \begin{itemize}
% \item
% $\| \V \| =1$
% \item 
% Example = $(\X,y)$, $y \in \{-1,+1\}$.
% \item
% $\forall \X,\;\; \| \X \| \leq R$.
% \item
% $\forall (\X,y),$\\$y(\X \cdot \V) \geq g$
% \end{itemize}
% \end{column}
% \end{columns}

% \end{frame}

% \item 
% Total Cost of Ownership (TCO) = datacenter depreciation + datacenter Opex + server depreciation + server Opex
% \item
% datacenter depreciation = chost of building the facility, amortied over
% the lifetime of facility (15 years).
% \item
% datacenter opex = cooling, maintenance and security of the facility
% \item
% server depreciation = cost of computer amortized over the lifetime of
% the server (3 years)
% \item
% server opex = electricity, computer maintenance, communication.

% \fi

\begin{frame}
\frametitle{Goals of the course}
\begin{itemize}
\item The hedging  / regret minimization framework.
\item A few hedging algorithms, with proofs.
\item A more theoretical treatment available in books, many papers,
  and previous course (available online).
\item Applications:
\begin{enumerate}
\item Computer Resource management.
\item Estimation, Tracking and control.
\item Financial portfolio management.
\item Many many more...
\end{enumerate}
\item \blue{This class:} An introduction through a resource management example.
\end{itemize}
\end{frame}

\section{managing contention through cost}

\begin{frame}
\frametitle{Resource sharing in computation}
\begin{itemize}
\item Traditional resources:
\begin{enumerate}
\item Run time.
\item Memory.
\end{enumerate}
\item Emerging resources:
\begin{enumerate}
\item Energy consumption.
\item Cooling.
\item Communication bandwidth.
\end{enumerate}
\end{itemize}
\end{frame}

\begin{frame}
\frametitle{The changing landscape}
\begin{itemize}
\item Moore's law
\begin{enumerate}
\item Doubling of clock speed: stopped around 2000.
\item Doubling of number of transistors: continuing.
\end{enumerate}
\item Parallelism is the only way forward
\begin{enumerate}
\item Multi-core.
\item Datacenters instead of super-computers.
\end{enumerate}
\item Computation as a commodity
\begin{enumerate}
\item From the Lab, to the Office, to the home.
\item Movement from one-of(protein folding) to scalable applications
  (google maps).
\end{enumerate}
\end{itemize}
\end{frame}

\begin{frame}
\frametitle{Resource allocation / the standard approach}
\begin{itemize}
\item Several tasks need to use a common resource.
\item The {\color{blue} scheduler} arbitrates use of resource.
\item Conflicts are resolved based on {\color{blue} priorities}
\begin{itemize}
\item High priority tasks complete before low priority tasks.
\item Tasks with the same priority wait in a queue.
\item Example: Fatal error > user input > computation > Housekeeping. 
\end{itemize}
\item Problems with priorities:
\begin{itemize}
\item Who assigns priorities? (Unix ``nice'')
\item Low priority queues can get starved. (``Your call will be
  answered in the order in which it was recieved.'')
\item No deadline guarantees (real-time operating systems).
\item Very hard to scale up to large distributed systems.
\end{itemize}
\end{itemize}
\end{frame}

\begin{frame}
\frametitle{An economic model of resource allocation}
\begin{itemize}
\item Assign price to each resource.
\item Assign revenue to the completion of each task.
\item Scheduler task is to maximize \\
{\color{blue} profit = total revenues - total costs.}
\item Example: scheduling search tasks on a 4-core machine.
\begin{itemize}
\item \blue{Revenue:} search completed < 0.5 sec -- {\textcent}0.1
\item \blue{Total cost:}  {\textcent}10 per hour.
\item \blue{Scheduling heuristic}: Schedule tasks in order
  recieved, occupying all available cores.
\end{itemize}
\end{itemize}
\end{frame}

\begin{frame}
\frametitle{Contension resolution through prices}
\begin{itemize}
\item Profit analysis
\begin{itemize}
\item Assume average search taked 0.1 sec on a single core.
\item Max throughput: 40 queries per second.
\item Max Revenue: {\textcent}2 per second. Break even after 5 seconds / hour. 
\end{itemize}
\item {\bf bursty load:} we get 1000 queries in 10 seconds and nothing
  the rest of the hour.
\begin{itemize}
\item Potential revenue: \$1.
\item Revenue with limited resources: {/textcent}20
\item Better to use $5 \times 4=20$ cores for 10 seconds and nothing
  for the rest of the hour.
\end{itemize}
\item Can we benefit from computation on the cloud?
\end{itemize}
\end{frame}

\begin{frame}
\frametitle{Amazon's Elastic cloud (EC2)}
\begin{itemize}
\item Small instance, Windows OS, North California
% #  Break even point is 28% utilization. 
\begin{itemize}
\item \blue{cost} {\textcent}13 per hour 
\item \$93.6 per month if used 100\% of time.
\end{itemize}
\item Reserving computers.
\begin{itemize}
\item Paying \$227.5 per year reduces the price to {\textcent}4 per
  hour.
\item Break even point is 28\%
\item If load in coming year > 28\% - reservation is worthwhile.
\end{itemize}
\item Spot pricing - a market system.
\end{itemize}
\end{frame}

\begin{frame}
\frametitle{The spot market in EC2}
\multiinclude[graphics={width=4in},format=pdf]{SpotMarket}
\end{frame}

\section{Example: putting computer screen to sleep}

\begin{frame}
\frametitle{Simpler problem: putting screen to sleep}
\begin{itemize}
\item Many variants: energy management in data-center, WiFi link, hard-disk spindown.
\item Standard heuristic: timeout.
\item Failures: screen turns off while watching a movie, giving a talk
  on the computer.
\item Underlying problem: predicting when screen will be turned back
  on (user keystroke).
\end{itemize}
\end{frame}

\begin{frame}
\frametitle{Relevant information for prediction}
\begin{itemize}
\item time since last keystroke.
\item number of keystrokes in last minute, 10 minutes, hour ...
\item active application (powerpoint, DVD player, skype ...)
\item detection of user through microphone, video camera ...
\end{itemize}
\end{frame}
 
\begin{frame}
\frametitle{Heuristics / Experts}
\begin{itemize}
\item Expert = a rule for deciding when to turn off screen.
\item \blue{Timeout Expert}: Keyboard idle for 5 minutes.
\item \blue{Relative Timeout Expert}: Kbd idle $\geq$ \newline
  ($C \times$ average time between keystrokes in last hour)
\item \blue{Night Expert:} Kbd idle 30 minutes after 6pm.
\item \blue{App Expert}: Keep on $2$ hours if DVD is playing.
\item \blue{Sensor Expert}: Don't turn off if detecting user watching screen.
\end{itemize}
\end{frame}

\begin{frame}
\frametitle{Hedging}
\begin{itemize}
\item \blue{\bf Hedge:} A {\bf master algorithm} that combines the experts.
\item Each Expert is assigned a {\bf weight}.
\begin{itemize}
\item Large weight - expert is performing well - use this expert more.
\item Small weight - expert performing poorly - use this expert less.
\end{itemize} 
\item \blue{\bf Goal of Hedge:} Perform almost as well as the best expert
  in hind-sight.
\item \blue{\bf Performance measure:} cumulative gain/loss.
\end{itemize}
\end{frame}

\begin{frame}
\frametitle{Loss/gain for single screen turn-off}
\multiinclude[graphics={width=3.5in},format=pdf]{Figures/Cumulative}
\end{frame}

\iffalse %%%%%%%%%%%%%%%%%%%%%%%%%%%%%%%%%%%%%%%%%%%%%%%%%%%%%%%%%%%%%%%%
\fi
