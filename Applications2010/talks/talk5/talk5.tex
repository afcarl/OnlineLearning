%\documentclass{beamer}
\documentclass[handout]{beamer}

\mode<presentation>
{
  \usetheme{Montpellier}

  %\setbeamercovered{transparent}
  % or whatever (possibly just delete it)
}

\usepackage{xmpmulti} % package that defines \multiinclude
\usepackage[english]{babel}
\usepackage[latin1]{inputenc}

\usepackage{times}
\usepackage[T1]{fontenc}

% Or whatever. Note that the encoding and the font should match. If T1
% does not look nice, try deleting the line with the fontenc.

\title[\ouralg] % (optional, use only with long paper titles)
{Game Theory, online learning and boosting}

\author[Freund] % (optional, use only with lots of authors)
{Yoav Freund}
% - Give the names in the same order as the appear in the paper.
% - Use the \inst{?} command only if the authors have different
%   affiliation.

\institute[Universities of Somewhere and Elsewhere] % (optional, but mostly needed)

\subject{Machine Learning}
% This is only inserted into the PDF information catalog. Can be left
% out. 

% If you have a file called "university-logo-filename.xxx", where xxx
% is a graphic format that can be processed by latex or pdflatex,
% resp., then you can add a logo as follows:

% \pgfdeclareimage[height=0.5cm]{university-logo}{university-logo-filename}
% \logo{\pgfuseimage{university-logo}}



% Delete this, if you do not want the table of contents to pop up at
% the beginning of each subsection:
%% \AtBeginSubsection[]
%% {
%%   \begin{frame}<beamer>
%%     \frametitle{Outline}
%%     \tableofcontents[currentsection,currentsubsection]
%%   \end{frame}
%% }


% If you wish to uncover everything in a step-wise fashion, uncomment
% the following command: 

\beamerdefaultoverlayspecification{<+->}

\input{macros}

\begin{document}

\begin{frame}
  \titlepage
\end{frame}

\begin{frame}
  \frametitle{Outline}
  \tableofcontents[pausesections]
  % You might wish to add the option [pausesections]
\end{frame}

\section{The Minimax theorem}

\begin{frame}
\frametitle{Matrix Games}
\begin{center}
\begin{tabular}{ |c|c|c|c|c| }
\hline
  1 & 0 & 1 &0 &1\\
\hline
  -1 & 0 & 0 &1 & 1\\
\hline
  1 & 0 & -1 & 1 & 0\\
\hline
\end{tabular}
\end{center}

\begin{itemize}
\item A game between the \blue{column} player and the \blue{row}
  player.
\item The chosen entry defines the loss of column player = gain of row player.
\item If choices made serially, second player to choose has an advantage.
\end{itemize}
\end{frame}

\begin{frame}
\frametitle{Mixed strategies}
\begin{center}
\begin{tabular}{l|c|c|c|c|c|}
  & $q_1$ & $q_2$ & $q_3$ & $q_4$ & $q_5$ \\
\hline
$p_1$ &  1 & 0 & 1 &0 &1\\
\hline
$p_2$ &  -1 & 0 & 0 &1 & 1\\
\hline
$p_3$ &  1 & 0 & -1 & 1 & 0\\
\hline
\end{tabular}
\end{center}

\begin{itemize}
\item \blue{pure} strategies: each player chooses a single action.
\item \blue{mixed} strategies: each player chooses a distribution over actions.
\item Expeted gain/loss: \redEq{\vec{p} M \vec{q}^T}
\end{itemize}
\end{frame}

\begin{frame}
\frametitle{The Minimax theorem}

John Von-Neumann, 1928\newline

\red{
\[
\max_{\vec{p}} \min_{\vec{q}} \vec{p} M \vec{q}^T
=
\min_{\vec{q}} \max_{\vec{p}} \vec{p} M \vec{q}^T
\]
}

\begin{itemize}
\item Unlike pure strategies, the order of choice of mixed
  strategies does not matter.
\item \blue{Optimal mixed strateigies:} the strategies that achieve
  the minimax.
\item \blue{Value} of the game: the value of the minimax.
\item Finding the minimax strategies when the matrix is known = Linear Programming.
\end{itemize}
\end{frame}

\section{Learning games}

\begin{frame}
\frametitle{A matrix corresponding to online learning.}
\begin{center}
\begin{tabular}{l|c|c|c|c|c|}
  $t=$   & 1  & 2 & 3 & 4 & ... \\
\hline
expert 1 & 1  & 0 & 1 & 0 & ...\\
\hline
expert 2 & -1 & 0 & 0 & 1 & ...\\
\hline
expert 3 &  1 & 0 & -1 & 1 & ...\\
\hline
\end{tabular}
\end{center}

\begin{itemize}
\item The columns are revealed one at a time.
  strategies does not matter.
\item Using Hedge or NormalHedge the row player chooses a mixed
  strategy over the rows that is almost as good as the best single row
  in hind-sight.
\item The best single row in hind-site is at least as good as any
  mixed strategy in hind-sight.
\end{itemize}
\end{frame}

\begin{frame}
\frametitle{A matrix corresponding to online learning.}
\begin{center}
\begin{tabular}{l|c|c|c|c|c|}
  $t=$   & 1  & 2 & 3 & 4 & ... \\
\hline
expert 1 & 1  & 0 & 1 & 0 & ...\\
\hline
expert 2 & -1 & 0 & 0 & 1 & ...\\
\hline
expert 3 &  1 & 0 & -1 & 1 & ...\\
\hline
\end{tabular}
\end{center}

\begin{itemize}
\item If the adversary playes optimally, then the row distribution
  coverges to a minimax optimal mixed strategy.
\item But adversary might not play optimally - minimizing regret is a
  stronger criterion than converging to minimax optimal mixed strategy.
\end{itemize}
\end{frame}

%% Insert slides with review of boosting.

\begin{frame}
\frametitle{A matrix corresponding to boosting}

\begin{center}
\begin{tabular}{l|c|c|c|c|c|}
    & ex. 1  & ex. 2 & ex. 3 & ex. 4 & ... \\
\hline
base rule 1 & 1  & 0 & 1 & 0 & ...\\
\hline
base rule 2 & 0 & 0 & 0 & 1 & ...\\
\hline
base rule 3 &  1 & 0 & 0 & 1 & ...\\
\hline
\end{tabular}
\end{center}

\begin{itemize}
\item \redEq{0} mistake, \redEq{1} correct.
\item A weak learning algorithm: can find a base rule whose weighted error is
  smaller than \redEq{1/2-\gamma} or \blue{any} distribution over the examples.
\item There is a distribution over the base rules such that for any
  example the expected error is smaller \redEq{1/2-\gamma}.
\item Implies that the majority vote wrt this distribution over base
  rules is correct on {\bf all} examples.
\item Moreover - the weight of the majority is at least
  \redEq{1/2+\gamma}, the minority is at most \redEq{1/2-\gamma}.
\end{itemize}

\end{frame}

\iffalse

\begin{frame}
\frametitle{Using \ouralg for tracking}
\begin{itemize}
\item Use exponentially discounted loss.
\item \ouralg guaranteed to perform almost as well as best expert
  \blue{with respect to exp. discounted loss}.
\item Tracks state well when changes occur every \redEq{1/\alpha}
  examples.
\item Choosing the learning rate \redEq{\eta} is a significant problem.
\end{itemize}
\end{frame}

\begin{frame}
\frametitle{Tracking using a noisy echo}
\multiinclude[graphics={width=4in},format=png]{figures/Noise}
\end{frame}

\begin{frame}
\frametitle{Tracking for \redEq{\sigma=1}}
\multiinclude[graphics={width=4in},format=png]{figures/sigma1}
\end{frame}

\begin{frame}
\frametitle{Tracking for \redEq{\sigma=2}}
\multiinclude[graphics={width=4in},format=png]{figures/sigma2}
\end{frame}

\begin{frame}
\frametitle{Tracking for \redEq{\sigma=4}}
\multiinclude[graphics={width=4in},format=png]{figures/sigma4}
\end{frame}

\begin{frame}
\frametitle{Tracking for \redEq{\sigma=8}}
\multiinclude[graphics={width=4in},format=png]{figures/sigma8}
\end{frame}
\fi

\end{document}


