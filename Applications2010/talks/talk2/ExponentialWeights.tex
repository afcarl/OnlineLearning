%\documentclass{beamer}
\documentclass[handout]{beamer}

\mode<presentation>
{
  \usetheme{Montpellier}

  %\setbeamercovered{transparent}
  % or whatever (possibly just delete it)
}

\usepackage{xmpmulti} % package that defines \multiinclude
\usepackage[english]{babel}
\usepackage[latin1]{inputenc}

\usepackage{times}
\usepackage[T1]{fontenc}

% Or whatever. Note that the encoding and the font should match. If T1
% does not look nice, try deleting the line with the fontenc.

\title[Exponential Weights Online Algorithms] % (optional, use only with long paper titles)
{Online Algorithms}

\author[Freund] % (optional, use only with lots of authors)
{Yoav Freund}
% - Give the names in the same order as the appear in the paper.
% - Use the \inst{?} command only if the authors have different
%   affiliation.

\institute[Universities of Somewhere and Elsewhere] % (optional, but mostly needed)

\subject{Machine Learning}
% This is only inserted into the PDF information catalog. Can be left
% out. 

% If you have a file called "university-logo-filename.xxx", where xxx
% is a graphic format that can be processed by latex or pdflatex,
% resp., then you can add a logo as follows:

% \pgfdeclareimage[height=0.5cm]{university-logo}{university-logo-filename}
% \logo{\pgfuseimage{university-logo}}



% Delete this, if you do not want the table of contents to pop up at
% the beginning of each subsection:
%% \AtBeginSubsection[]
%% {
%%   \begin{frame}<beamer>
%%     \frametitle{Outline}
%%     \tableofcontents[currentsection,currentsubsection]
%%   \end{frame}
%% }


% If you wish to uncover everything in a step-wise fashion, uncomment
% the following command: 

\beamerdefaultoverlayspecification{<+->}

\input{macros}

\begin{document}

\begin{frame}
  \titlepage
\end{frame}

\begin{frame}
  \frametitle{Outline}
  \tableofcontents[pausesections]
  % You might wish to add the option [pausesections]
\end{frame}

\section{Estimation}

\begin{frame}
\frametitle{The static discrete estimation problem}
\begin{itemize}
\item A system has an internal unobservable \blue{state}:
\redEq{s \in \{0,1,\ldots,K\}}
\item \blue{observations} are corrupted versions of the state: 
\redEq{o_1,o_2,\dots\;\; o_t \in \{0,1,\ldots,K\}}
\item Our goal is to estimate \redEq{s} from \redEq{o_1,o_2,\dots,o_T}.
\end{itemize}
\end{frame}

\begin{frame}
\frametitle{Estimation by maximizing likelihood}
\begin{itemize}
\item Prob. of observation \RedEq{o} conditioned on state \RedEq{s}:\newline
\RedEq{P\paren{O=o | S=s}}
\item Given \redEq{o_1,o_2,\dots,o_T} define likelihood of state
  \RedEq{s} as \RedEq{\prod_{t=1}^T P\paren{O=o_t | S=s}}.
\end{itemize}
\end{frame}
 
\begin{frame}
\frametitle{The static estimation problem}
\begin{itemize}
\item 
\end{itemize}
\end{frame}
\begin{frame}

\iffalse %%%%%%%%%%%%%%%%%%%%%%%%%%%%%%%%%%%%%%%
\fi

\end{document}


